\documentclass{article}
\begin{document}
There is no reason for us to believe that our current knowledge in particle physics is complete. The most
successful framework – the Standard Model – has notable shortcomings. It doesn't explain dark matter.
Neither does it explain why the universe didn't get annihilated by anti-matter at the Big Bang.
To make matters
even worse, three of the massless particles have been observed to have mass. In this thesis, we focus on
the said particles – the neutrinos – and propose extensions to the Standard Model related to them. We
then compare the effects of these extensions with collected data to see if our new theory more closely
describes Nature or not.
Ultimately, the purpose of this excursion is to guide future research where to uncover more accurate
theories. Traces of this grander theory will be present as breadcrumbs scattered in Nature. We just need
to know where and how carefully to look.

In order to describe the three quantizable forces (interactions) of nature, we gather the mediators of each force – the
vector bosons – into local symmetry groups called gauge groups. Each vector boson has one corresponding generator,
the set of which constitutes the group. The strong charge is mediated by eight massless gluons, which
correspond to the eight independent generators of SU(3). The weak charge is mediated by the three massive gauge
bosons W and Z, and the massless photon γ, which constitute the generators of SU(2) and U(1).
The subscript of each group denotes by which mechanism that force is mediated. The gluons mediate
the strong force through interactions of color, emphasized with subscript C. The weak force only sees
left-handed particles, which we distinguish with the subscript L. And the electroweak interaction that
a particle undergoes is determined by its hypercharge Y. For example, the quarks all have a nonzero
color and hypercharge, so they participate in the strong and electromagnetic interactions. If a quark is
left-handed, it will also feel the weak interaction. The neutrinos on the other hand, have neither charge
nor color, so they are insivible to both the strong and electromagnetic force. We express this by letting
their fields transform as singlets under those SU(3) and U(1).
Together, these three interactions make up the Standard Model gauge group SU(3)_C × SU(2)_L ×
U(1)_Y.

TODO: NEUTRINO OSCILLATIONS AND MATTER EFFECTS 


We always observe neutrinos indirectly through their associated charged lepton. Regardless of the type
of interaction (charged current via the W boson, or neutral current via the Z), a charged lepton exits
with altered properties. The lepton is then detected, and the properties of the neutrino involved in the
interaction is then deduced. This deduction is obviously imperfect.

TODO: ABOUT ICECUBE
In this work, we only study the detectors handeled by the IceCube collaboration.
They are of
Cherenkov type, which means that they detect the secondary charged lepton by its emitted Cherenkov
light, produced from its travel through the Antarctic ice. If the charged leptons interact heavily with the
ice, they will travel a short distance and emit a localized flash of Cherenkov light. This event is referred
to as a cascade.
If the charged leptons don’t interact as much in the ice, they penetrate a larger part of it,
emitting light and tertiary particles as they go. This event is referred to as a track, and are often due to
muon charged current interactions.
After an event has occurred, the IceCube algorithms process the data coming from the detector to
reconstruct the event. This means that, given the parameters recorded by the detector, what are their
”true” values? We are interested in two variables: the energy and the direction. Each event is tagged with
a probable energy and zenith angle, called the recostructed parameters Ereco and cos (θrecoz), which are
the parameters according to the DOMs. The collaboration then uses numerous sophisticated methods to
backtrack the reconstructed parameters to the true parameters. So a charged lepton hits the DOMs, and
we ultimately end up with the associated neutrino’s true and reconstructed energy and zenith angle. The
reconstructed parameters are what we are using to analyze the data (because this is what the detector
actually sees), while the true parameters are used in the determination of that neutrino’s ‘actual’ flux
and cross-section (because this is what Nature sees).

As the neutrinos have propagated the Earth, they arrive at the South Pole, where they interact with
charged lepton in the ice. We now are interested in the effective area Aeff, i.e. the cross-section of the
detector that the lepton is exposed to. Aeff depends on several parameters, some of them being detector
physical volume, Etrue, cos(θtruez) and the neutrino cross-section.

So now we have the physical quantities in the true parameters. But as we discussed, we need a way to
translate this into the reconstructed parameters that the detector gives us. We will call the relationship
between Ereco and Etrue the energy resolution function, and the relationshop between cos (θrecoz) and
cos (θtruez) the zenith resolution function. We assume the relationship to follow a logarithmic Gaussian
distribution, giving it the form ....

So how do we then obtain the parameters needed to construct the Gaussian?
The collaboration frequently releases their own simulated data files, which we refer to the IceCube Monte Carlo.
In the files, we can see the connection between the true and reconstructed parameters. We use this to train a Gaussian
Process Regressor.
TODO: ABOUT GPR

Independent researchers outside of the IceCube collaboration will not be able to more persicely simulate
the detector. The IceCube Monte Carlo is a complex and proprietary machinery, so our goal in this
section is to come as close as we can to their Monte Carlo simulations. After we are confident that our
code displays the same overall features as the ‘offical’, we normalize our results to it.
For each bin i, j, we then obtain a correction factor which contains information that we are unable to
obtain or sufficiently incorporate. One example of such information is the systematic errors of the DOMs.
Recent IceCube data releases do not include such information. Since the systematic errors are affecting
the event count on a bin-by-bin basis, they can in theory drastically modify the binned results. Another
example of an error source what will be remedied by this method is the flux. We are using a fairly simple
model of the atmospheric flux that excludes atmospheric prompt and astrophysical fluxes. The IceCube
collaboration use several different flux models which are initialized by a parametrization of the cosmic
ray flux. Included in the cosmic ray models are e.g. the pion to kaon ratio, which are often used as a nuisance parameter. By not
being able to include this in our error analysis, our method will be limited to only consider the overall flux normalization,
rather than the components that produce the flux in the first place.

In 1996, the LSND experiment reported an excess of ¯νe events from an ¯νµ beam [17]. Nine years later,
MiniBooNE not only reproduced the ¯νe anomaly, but observed an excess in the νe events too. Together
with the so-called reactor and gallium anomalies, these reports suggested that the three massive neutrino
framework could be amended.
By introducing a fourth heavy mass state ν4 at eV scale, both appearance and
disappeareance anomalies could be minimally accomodated. However, we know from the decay width
of the Z boson that it only can interact with three flavor species1, so this fourth mass state cannot be
interacting weakly. In other words, it needs to transform as a singlet under the broken gauge group
SU(2)L × U(1)EM. We now distinguish between the three original neutrino flavors (e, µ, and τ) and the
new fourth flavor by calling the former active neutrinos and the latter sterile due to its non-interacting
behavior.

Since the new neutrino does not interact weakly, how do we then detect its signal? If the sterile mixing
angle θi4 is non-zero, we allow the sterile mass state to mix with the active state i. The most interesting
case is when θ24 ̸= 0, which for ∆m2
41 ∼ eV2 gives rise to a resonant disappeareance in the for P¯νµ¯νµ


We have no reason to believe that the Standard Model gauge group is the complete picture. We have
already seen how the failure of the Standard Model to predict neutrino masses and flavor osillations forces
us to amend it. Just as the electroweak theory SU(2)L × U(1)Y is spontaneously broken to U(1)EM,
a higher order theory at a different energy scale with completely different properties might undergo a
similar spontaneous symmetry break at high energies, producing at lower energies the Standard Model
that we know. In fact, just as the fermion masses originate from the electroweak symmetry breaking,
the neutrino masses might be generated from another broken symmetry resulting in the Standard Model
gauge group. In this sense, the Standard Model might be considered an effective low energy theory,
which is yiels impressive results in some areas, but failing in others.

Compared to other fermions, the lightness of neutrino masses4 along with their sparse SM interactions
might indicate that these particles provide the best starting point for us to probe new physics, in which
exotic neutrino interactions might occur [21]. We call these interactions non-standard interactions (NSI)
in order to distinguish them from the ‘standard’ interactions of the Standard Model.
Following the approach of the Standard Model that the group generators uniquely determine the gauge
bosons, a different gauge theory will have different interactions than those we presently know. These
new interactions can be parametrized as model-independent four-fermion effective operators [22, 23].
Following the discussion in [24], we see that the NSI parameters ϵ resulting from six-dimensional operators
have the scale ...

Up until now, we have only considered weak neutrino interactions with electrons, protons, and neutrons.
We can phenomenologically allow these interactions to include the up and down quarks which are present
in the Earth as the fundamental components of neutrons and protons.
The CC NSI effect affects neutrino production and
detection, and will not be considered here. NC NSI affect the matter potential, and is thus of interest
to us.

Off-iagonal NSI parameters constitute new sources of flavor violation, while diagonal elements represent 
flavor nonuniversality.

In our analysis of IceCube, we are constrained to muon track events. Thus, we are not able to test any
theory which does not modify Pαµ. Moreover, the IceCube data is avaibable in the range 500 GeV to
10 TeV range, where any rapid oscillations have settled down.

\end{document}