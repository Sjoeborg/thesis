% \documentclass[draft=True]{thesis}
% \usepackage[margin=0.5in]{geometry}
% \usepackage{graphicx, slashed, siunitx}
% \usepackage[utf8]{inputenc}
% \usepackage{xr-hyper}
% \documentclass[a4paper,10pt,draft]{thesis}
\usepackage{physics,amsmath, amsfonts, siunitx, amssymb, graphicx, slashed,subcaption}
\usepackage[utf8]{inputenc}
\usepackage[margin=1in]{geometry}
\usepackage[hidelinks]{hyperref}
\usepackage{xr-hyper}
\newcommand{\n}[1]{\nu_{#1}}
\newcommand{\na}{\nu_\alpha}
\newcommand{\nb}{\nu_\beta}
\newcommand{\ana}{\bar{\nu}_\alpha}
\newcommand{\an}[1]{\bar{\nu}_{\text{#1}}}
\newcommand{\anb}{\bar{\nu}_\beta}
\renewcommand{\a}{\alpha}
\renewcommand{\b}{\beta}
\newcommand{\ab}{\alpha\beta}


\renewcommand{\ne}{\nu_e}
\newcommand{\nm}{\nu_\mu}
\newcommand{\nt}{\nu_\tau}
\newcommand{\ns}{\nu_s}

\newcommand{\ane}{\bar{\nu}_e}
\newcommand{\anm}{\bar{\nu}_\mu}
\newcommand{\ant}{\bar{\nu}_\tau}
\newcommand{\ans}{\bar{\nu}_s}

\newcommand{\nee}{\nu_e \to \nu_e}
\newcommand{\nem}{\nu_e \to \nu_\mu}
\newcommand{\net}{\nu_e \to \nu_\tau}
\newcommand{\nes}{\nu_e \to \nu_s}

\newcommand{\nme}{\nu_\mu \to \nu_e}
\newcommand{\nmm}{\nu_\mu \to \nu_\mu}
\newcommand{\nmt}{\nu_\mu \to \nu_\tau}
\newcommand{\nms}{\nu_\mu \to \nu_s}



\newcommand{\Pee}{P_{e  e}}
\newcommand{\Pem}{P_{e  \mu}}
\newcommand{\Pet}{P_{e  \tau}}
\newcommand{\Pes}{P_{e  s}}

\newcommand{\Pme}{P_{\mu  e}}
\newcommand{\Pmm}{P_{\mu\mu}}
\newcommand{\Pmt}{P_{\mu  \tau}}
\newcommand{\Pms}{P_{\mu  s}}


\newcommand{\Pte}{P_{P_{\tau e}}}
\newcommand{\Ptm}{P_{\tau  \mu}}
\newcommand{\Ptt}{P_{\tau  \tau}}
\newcommand{\Pts}{P_{\mu  s}}

\newcommand{\Paeae}{P_{\bar{e}  \bar{e}}}
\newcommand{\Paeam}{P_{\bar{e}  \bar{\mu}}}
\newcommand{\Paeat}{P_{\bar{e}  \bar{\tau}}}
\newcommand{\Paeas}{P_{\bar{e}  \bar{s}}}

\newcommand{\Pamae}{P_{\bar{\mu}  \bar{e}}}
\newcommand{\Pamam}{P_{\bar{\mu}  \bar{\mu}}}
\newcommand{\Pamat}{P_{\bar{\mu}  \bar{\tau}}}
\newcommand{\Pamas}{P_{\bar{\mu}  \bar{s}}}


\newcommand{\Patae}{P_{\bar{\tau}  \bar{e}}}
\newcommand{\Patam}{P_{\bar{\tau}  \bar{\mu}}}
\newcommand{\Patat}{P_{\bar{\tau}  \bar{\tau}}}
\newcommand{\Patas}{P_{\bar{\mu}  \bar{s}}}

\renewcommand{\th}[1][]{%
  \theta\ifx\\#1\\\else_\text{#1}\fi
}
\newcommand{\thm}[1][]{%
  \theta^\text{M}\ifx\\#1\\\else_\text{#1}\fi
}
\renewcommand{\t}[1]{\text{{#1}}}
\newcommand{\avg}[1]{\left\langle {#1} \right \rangle}
\newcommand*{\dm}[1][]{%
  \Delta m^2\ifx\\#1\\\else_\text{#1}\fi
}
\newcommand{\zreco}{\cos{(\theta_z^{reco})}}
\newcommand{\ztrue}{\cos{(\theta_z^{true})}}
\newcommand{\z}{\cos{(\theta_z)}}
\newcommand{\Ereco}{E^{reco}}
\newcommand{\Etrue}{E^{true}}
\newcommand{\Aeff}{A^\text{eff}}
\newcommand{\emm}{\epsilon_{\mu\mu}}
\newcommand{\emt}{\epsilon_{\mu\tau}}
\newcommand{\eet}{\epsilon_{e\tau}}
\newcommand{\eem}{\epsilon_{e\mu}}
\newcommand{\ett}{\epsilon_{\tau\tau}}
\newcommand{\ep}{\epsilon^\prime}

% \begin{document}
\section{Non-standard Interactions}
\subsection{$\chi^2$ minimization}
The $\chi^2$ takes the same form as in Eq.~\ref{eq:chisq}, namely 
\begin{align} \label{eq:chisq2}
    \chi^{2}(\hat{\theta},\alpha,\beta)=\sum_{ij} \frac{\left(N^\text{th}-N^\text{data}\right)_{ij}^{2}}
    {\left(\sigma^\text{data}_{ij}\right)^{2} + \left(\sigma^\text{syst}_{ij}\right)^{2}}+ 
    \frac{(1-\alpha)^2}{\sigma_\alpha^2} + \frac{\beta^2}{\sigma_\beta^2}\,
\end{align}
Just as with the sterile analysis, the IceCube event count takes the form 
\begin{align}
    N^\text{th}_{ij} = \alpha\left[1+\beta (0.5 + \zreco_i )\right] N_{ij}(\hat{\theta})\,.
\end{align}
For DeepCore and PINGU however, the event count takes the form
\begin{align}
    N^\text{th}_{ijk} = \alpha\left[1+\beta \zreco_i \right] N_{ijk}(\hat{\theta}) + \kappa N_{ijk}^{\mu_{atm}}\,,
\end{align}
with $N_{ijk}(\hat{\theta})$ from Eq.~\ref{eq:MCevents}. $N_{ijk}^{\mu_{atm}}$ is the muon background, which is left to float freely in the DeepCore analysis.
The background at PINGU can be considered neglible to first order~\cite{PINGUdata}, and we thus put $\kappa=0$ when calculating the PINGU $\chi^2$\,.
For IceCube, we set $\sigma_{ijk}^\text{syst} = f\sqrt{N_{ijk}^\text{data}}$.
For DeepCore, we use the provided systematic error distribution which accounts for both uncertanties in the finite MC statistics and in the data-driven 
muon background estimate~\cite{DC2019data}.
\subsection{Constraining Parameters}

% \end{document}