% \documentclass[draft=True]{thesis}
% \usepackage[margin=0.5in]{geometry}
% \usepackage{graphicx, slashed, siunitx}
% \usepackage[utf8]{inputenc}
% \usepackage{xr-hyper}
% \usepackage{physics,amsmath, amsfonts, siunitx, amssymb}
\usepackage[utf8]{inputenc}
\newcommand{\n}[1]{\ensuremath{\nu_{#1}}}
\newcommand{\na}{\ensuremath{\nu_\alpha}}
\newcommand{\nb}{\ensuremath{\nu_\beta}}
\newcommand{\ana}{\ensuremath{\bar{\nu}_\alpha}}
\newcommand{\an}[1]{\ensuremath{\bar{\nu}_{\text{#1}}}}
\newcommand{\anb}{\ensuremath{\bar{\nu}_\beta}}
\renewcommand{\a}{\ensuremath{\alpha}}
\renewcommand{\b}{\ensuremath{\beta}}
\newcommand{\ab}{\ensuremath{\alpha\beta}}

\renewcommand{\ne}{\ensuremath{\nu_e}}
\newcommand{\ns}{\ensuremath{\nu_s}}

\newcommand{\nee}{\ensuremath{\nu_e \to \nu_e}}
\newcommand{\nem}{\ensuremath{\nu_e \to \nu_\mu}}
\newcommand{\net}{\ensuremath{\nu_e \to \nu_\tau}}
\newcommand{\nes}{\ensuremath{\nu_e \to \nu_s}}

\newcommand{\nme}{\ensuremath{\nu_\mu \to \nu_e}}
\newcommand{\nmm}{\ensuremath{\nu_\mu \to \nu_\mu}}
\newcommand{\nmt}{\ensuremath{\nu_\mu \to \nu_\tau}}
\newcommand{\nms}{\ensuremath{\nu_\mu \to \nu_s}}


\newcommand{\nte}{\ensuremath{\nu_\tau \to \nu_e}}
\newcommand{\ntm}{\ensuremath{\nu_\tau \to \nu_\mu}}
\newcommand{\ntt}{\ensuremath{\nu_\tau \to \nu_\tau}}
\newcommand{\nts}{\ensuremath{\nu_\mu \to \nu_s}}

\newcommand{\Pee}{\ensuremath{P_{e  e}}}
\newcommand{\Pem}{\ensuremath{P_{e  \mu}}}
\newcommand{\Pet}{\ensuremath{P_{e  \tau}}}
\newcommand{\Pes}{\ensuremath{P_{e  s}}}

\newcommand{\Pme}{\ensuremath{P_{\mu  e}}}
\newcommand{\Pmm}{\ensuremath{P_{\mu\mu}}}
\newcommand{\Pmt}{\ensuremath{P_{\mu  \tau}}}
\newcommand{\Pms}{\ensuremath{P_{\mu  s}}}


\newcommand{\Pte}{\ensuremath{P_{P_{\tau e}}}}
\newcommand{\Ptm}{\ensuremath{P_{\tau  \mu}}}
\newcommand{\Ptt}{\ensuremath{P_{\tau  \tau}}}
\newcommand{\Pts}{\ensuremath{P_{\mu  s}}}

\newcommand{\Paeae}{\ensuremath{P_{\bar{e}  \bar{e}}}}
\newcommand{\Paeam}{\ensuremath{P_{\bar{e}  \bar{\mu}}}}
\newcommand{\Paeat}{\ensuremath{P_{\bar{e}  \bar{\tau}}}}
\newcommand{\Paeas}{\ensuremath{P_{\bar{e}  \bar{s}}}}

\newcommand{\Pamae}{\ensuremath{P_{\bar{\mu}  \bar{e}}}}
\newcommand{\Pamam}{\ensuremath{P_{\bar{\mu}  \bar{\mu}}}}
\newcommand{\Pamat}{\ensuremath{P_{\bar{\mu}  \bar{\tau}}}}
\newcommand{\Pamas}{\ensuremath{P_{\bar{\mu}  \bar{s}}}}


\newcommand{\Patae}{\ensuremath{P_{\bar{\tau}  \bar{e}}}}
\newcommand{\Patam}{\ensuremath{P_{\bar{\tau}  \bar{\mu}}}}
\newcommand{\Patat}{\ensuremath{P_{\bar{\tau}  \bar{\tau}}}}
\newcommand{\Patas}{\ensuremath{P_{\bar{\mu}  \bar{s}}}}

\renewcommand{\th}[1][]{%
  \theta\ifx\\#1\\\else_\text{#1}\fi
}
\newcommand{\thm}[1][]{%
  \theta^\text{M}\ifx\\#1\\\else_\text{#1}\fi
}
\renewcommand{\t}[1]{\ensuremath{\text{{#1}}}}
\newcommand{\avg}[1]{\ensuremath{\left\langle {#1} \right \rangle}}
\newcommand*{\dm}[1][]{%
  \Delta m^2\ifx\\#1\\\else_\text{#1}\fi
}
\newcommand{\zreco}{\ensuremath{\cos{(\theta_z^{reco})}}}
\newcommand{\ztrue}{\ensuremath{\cos{(\theta_z^{true})}}}
\newcommand{\emm}{\ensuremath{\epsilon_{\mu\mu}}}
\newcommand{\emt}{\ensuremath{\epsilon_{\mu\tau}}}
\newcommand{\eet}{\epsilon_{e\tau}}
\newcommand{\eem}{\epsilon_{e\mu}}
\newcommand{\ett}{\ensuremath{\epsilon_{\tau\tau}}}
\newcommand{\ep}{\ensuremath{\epsilon^\prime}}
\renewcommand{\ne}{\nu_e}
\newcommand{\nm}{\nu_\mu}
\newcommand{\nt}{\nu_\tau}
\newcommand{\ane}{\bar\nu_e}
\newcommand{\anm}{\bar\nu_\mu}
\newcommand{\ant}{\bar\nu_\tau}
% \begin{document}
\section{Non-standard Interactions}
\subsection{$\chi^2$ minimization}
The $\chi^2$ takes the same form as in Eq.~\ref{eq:chisq}, namely 
\begin{align} \label{eq:chisq2}
    \chi^{2}(\hat{\theta},\alpha,\beta)=\sum_{ij} \frac{\left(N^\text{th}-N^\text{data}\right)_{ij}^{2}}
    {\left(\sigma^\text{data}_{ij}\right)^{2} + \left(\sigma^\text{syst}_{ij}\right)^{2}}+ 
    \frac{(1-\alpha)^2}{\sigma_\alpha^2} + \frac{\beta^2}{\sigma_\beta^2}\,
\end{align}
Just as with the sterile analysis, the IceCube event count takes the form 
\begin{align}
    N^\text{th}_{ij} = \alpha\left[1+\beta (0.5 + \zreco_i )\right] N_{ij}(\hat{\theta})\,.
\end{align}
For DeepCore and PINGU however, the event count takes the form
\begin{align}
    N^\text{th}_{ijk} = \alpha\left[1+\beta \zreco_i \right] N_{ijk}(\hat{\theta}) + \kappa N_{ijk}^{\mu_{atm}}\,,
\end{align}
with $N_{ijk}(\hat{\theta})$ from Eq.~\ref{eq:MCevents}. $N_{ijk}^{\mu_{atm}}$ is the muon background, which is left to float freely in the DeepCore analysis.
The background at PINGU can be considered neglible to first order~\cite{PINGUdata}, and we thus put $\kappa=0$ when calculating the PINGU $\chi^2$\,.
For IceCube, we set $\sigma_{ijk}^\text{syst} = f\sqrt{N_{ijk}^\text{data}}$.
For DeepCore, we use the provided systematic error distribution which accounts for both uncertanties in the finite MC statistics and in the data-driven 
muon background estimate~\cite{DC2019data}.
\subsection{Constraining Parameters}
% \end{document}