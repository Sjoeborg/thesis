\documentclass[draft=True]{thesis}
\usepackage[margin=0.5in]{geometry}
\usepackage{graphicx, slashed, siunitx}
\usepackage[utf8]{inputenc}
\usepackage{xr-hyper}
\usepackage{physics,amsmath, amsfonts, siunitx, amssymb}
\usepackage[utf8]{inputenc}
\newcommand{\n}[1]{\ensuremath{\nu_{#1}}}
\newcommand{\na}{\ensuremath{\nu_\alpha}}
\newcommand{\nb}{\ensuremath{\nu_\beta}}
\newcommand{\ana}{\ensuremath{\bar{\nu}_\alpha}}
\newcommand{\an}[1]{\ensuremath{\bar{\nu}_{\text{#1}}}}
\newcommand{\anb}{\ensuremath{\bar{\nu}_\beta}}
\renewcommand{\a}{\ensuremath{\alpha}}
\renewcommand{\b}{\ensuremath{\beta}}
\newcommand{\ab}{\ensuremath{\alpha\beta}}

\renewcommand{\ne}{\ensuremath{\nu_e}}
\newcommand{\ns}{\ensuremath{\nu_s}}

\newcommand{\nee}{\ensuremath{\nu_e \to \nu_e}}
\newcommand{\nem}{\ensuremath{\nu_e \to \nu_\mu}}
\newcommand{\net}{\ensuremath{\nu_e \to \nu_\tau}}
\newcommand{\nes}{\ensuremath{\nu_e \to \nu_s}}

\newcommand{\nme}{\ensuremath{\nu_\mu \to \nu_e}}
\newcommand{\nmm}{\ensuremath{\nu_\mu \to \nu_\mu}}
\newcommand{\nmt}{\ensuremath{\nu_\mu \to \nu_\tau}}
\newcommand{\nms}{\ensuremath{\nu_\mu \to \nu_s}}


\newcommand{\nte}{\ensuremath{\nu_\tau \to \nu_e}}
\newcommand{\ntm}{\ensuremath{\nu_\tau \to \nu_\mu}}
\newcommand{\ntt}{\ensuremath{\nu_\tau \to \nu_\tau}}
\newcommand{\nts}{\ensuremath{\nu_\mu \to \nu_s}}

\newcommand{\Pee}{\ensuremath{P_{e  e}}}
\newcommand{\Pem}{\ensuremath{P_{e  \mu}}}
\newcommand{\Pet}{\ensuremath{P_{e  \tau}}}
\newcommand{\Pes}{\ensuremath{P_{e  s}}}

\newcommand{\Pme}{\ensuremath{P_{\mu  e}}}
\newcommand{\Pmm}{\ensuremath{P_{\mu\mu}}}
\newcommand{\Pmt}{\ensuremath{P_{\mu  \tau}}}
\newcommand{\Pms}{\ensuremath{P_{\mu  s}}}


\newcommand{\Pte}{\ensuremath{P_{P_{\tau e}}}}
\newcommand{\Ptm}{\ensuremath{P_{\tau  \mu}}}
\newcommand{\Ptt}{\ensuremath{P_{\tau  \tau}}}
\newcommand{\Pts}{\ensuremath{P_{\mu  s}}}

\newcommand{\Paeae}{\ensuremath{P_{\bar{e}  \bar{e}}}}
\newcommand{\Paeam}{\ensuremath{P_{\bar{e}  \bar{\mu}}}}
\newcommand{\Paeat}{\ensuremath{P_{\bar{e}  \bar{\tau}}}}
\newcommand{\Paeas}{\ensuremath{P_{\bar{e}  \bar{s}}}}

\newcommand{\Pamae}{\ensuremath{P_{\bar{\mu}  \bar{e}}}}
\newcommand{\Pamam}{\ensuremath{P_{\bar{\mu}  \bar{\mu}}}}
\newcommand{\Pamat}{\ensuremath{P_{\bar{\mu}  \bar{\tau}}}}
\newcommand{\Pamas}{\ensuremath{P_{\bar{\mu}  \bar{s}}}}


\newcommand{\Patae}{\ensuremath{P_{\bar{\tau}  \bar{e}}}}
\newcommand{\Patam}{\ensuremath{P_{\bar{\tau}  \bar{\mu}}}}
\newcommand{\Patat}{\ensuremath{P_{\bar{\tau}  \bar{\tau}}}}
\newcommand{\Patas}{\ensuremath{P_{\bar{\mu}  \bar{s}}}}

\renewcommand{\th}[1][]{%
  \theta\ifx\\#1\\\else_\text{#1}\fi
}
\newcommand{\thm}[1][]{%
  \theta^\text{M}\ifx\\#1\\\else_\text{#1}\fi
}
\renewcommand{\t}[1]{\ensuremath{\text{{#1}}}}
\newcommand{\avg}[1]{\ensuremath{\left\langle {#1} \right \rangle}}
\newcommand*{\dm}[1][]{%
  \Delta m^2\ifx\\#1\\\else_\text{#1}\fi
}
\newcommand{\zreco}{\ensuremath{\cos{(\theta_z^{reco})}}}
\newcommand{\ztrue}{\ensuremath{\cos{(\theta_z^{true})}}}
\newcommand{\emm}{\ensuremath{\epsilon_{\mu\mu}}}
\newcommand{\emt}{\ensuremath{\epsilon_{\mu\tau}}}
\newcommand{\eet}{\epsilon_{e\tau}}
\newcommand{\eem}{\epsilon_{e\mu}}
\newcommand{\ett}{\ensuremath{\epsilon_{\tau\tau}}}
\newcommand{\ep}{\ensuremath{\epsilon^\prime}}
\renewcommand{\ne}{\nu_e}
\newcommand{\nm}{\nu_\mu}
\newcommand{\nt}{\nu_\tau}
\newcommand{\ane}{\bar\nu_e}
\newcommand{\anm}{\bar\nu_\mu}
\newcommand{\ant}{\bar\nu_\tau}
\begin{document}
\section{DeepCore}
In this part, we use the publically available DeepCore data sample~\cite{DC2019data} which is an updated version of what was used by the 
IceCube collaboration in a $\nu_\mu$ disapprearance analysis~\cite{DC2018mudisappearance}.

The detector systematics include ice absorption and scattering, and overall, lateral, and head-on optical efficiencies of the DOMs. 
They are applied as correction factors using the best-fit points from the DeepCore 2019 $\nu_\tau$ appearance 
analysis~\cite{DC2019tauappearance}.

The data include 14901 track-like events and 26001 cascade-like events, both divided into eight 
$ \log_{10}E^{reco} \in [0.75,1.75]$ bins, and eight $\zreco \in [-1,1]$ bins. Each event has a Monte Carlo weight $w_{ijk,\beta}$,
from which we can construct the event count as
\begin{align}\label{eq:MCevents}
    N_{ijk} &= C_{ijk}\sum_{\beta}w_{ijk,\beta} \phi_\beta^\text{det}\,,
\end{align}
where $C_{k\beta}$ is the correction factor from the detector systematic uncertainty and $\phi_\beta^\text{det}$ is defined as Eq.~\ref{eq:propFlux}. We have now substituted the effect of the Gaussian smearing 
by treating the reconstructed and true quantities as a migration matrix. 

The oscillation parameters used on our DeepCore simulations are from the
best-fit in the global analysis in~\cite{nufit}: $\theta_{12} = \SI{33.44}{\degree},\, \theta_{13} = \SI{8.57}{\degree},\, \Delta m^2_{21} =  \SI{7.42}{\electronvolt^2}$, and we 
marginalize over $\dm$ and $\theta_{23}$.

We plot the event pull $(N_{NSI} - N_{SI})/\sqrt{N_{SI}}$ where $N_{(N)SI}$ are the numbers of expected events
assuming (non-)standard interactions in Fig.~\ref{fig:event_pulls}. This gives the normalized difference in the
number of expected events at the detector, and illustrates the expected sensitivity of DeepCore for the NSI parameters.
\section{PINGU}
The methodology behind the PINGU simulations are the same as with our DeepCore study~\ref{ch:DCmethod}. We use the public MC~\cite{PINGUdata}, which allows us to construct the event count as in Eq.~\ref{eq:MCevents}.
However, since no detector systematics is yet modelled for PINGU, the correction factors $C_{ijk}$ are all unity.
As with the DeepCore data, the PINGU MC is divided into eight 
$\log_{10}E^{reco} \in [0.75,1.75]$ bins, and eight $\zreco \in [-1,1]$ bins for both track- and cascade-like events. 
We generate "data" by predicting the event rates at PINGU with the following best-fit parameters from~\cite{nufit}, except for the CP-violating phase which is set to zero for simplicity.

\begin{align}\label{eq:PINGUparams}
    &\Delta m^2_{21} =  \SI{7.42e-5}{\electronvolt^2},\hspace{0.5em} \dm =  \SI{2.517e-3}{\electronvolt^2}, \nonumber \\
    &\theta_{12} = \SI{33.44}{\degree},\hspace{1em} \theta_{13} = \SI{8.57}{\degree},\hspace{1em} \theta_{23} = \SI{49.2}{\degree}, \hspace{1em} \delta_\text{CP} = 0\,.
\end{align}
\bibliographystyle{nature}
\bibliography{ref.bib}
\end{document}