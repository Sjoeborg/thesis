% \documentclass[draft=True]{thesis}
% \usepackage[margin=0.5in]{geometry}
% \usepackage{graphicx, slashed, siunitx}
% \usepackage[utf8]{inputenc}
% \usepackage{xr-hyper}
% \documentclass[a4paper,10pt,draft]{thesis}
\usepackage{physics,amsmath, amsfonts, siunitx, amssymb, graphicx, slashed,subcaption}
\usepackage[utf8]{inputenc}
\usepackage[margin=1in]{geometry}
\usepackage[hidelinks]{hyperref}
\usepackage{xr-hyper}
\newcommand{\n}[1]{\nu_{#1}}
\newcommand{\na}{\nu_\alpha}
\newcommand{\nb}{\nu_\beta}
\newcommand{\ana}{\bar{\nu}_\alpha}
\newcommand{\an}[1]{\bar{\nu}_{\text{#1}}}
\newcommand{\anb}{\bar{\nu}_\beta}
\renewcommand{\a}{\alpha}
\renewcommand{\b}{\beta}
\newcommand{\ab}{\alpha\beta}


\renewcommand{\ne}{\nu_e}
\newcommand{\nm}{\nu_\mu}
\newcommand{\nt}{\nu_\tau}
\newcommand{\ns}{\nu_s}

\newcommand{\ane}{\bar{\nu}_e}
\newcommand{\anm}{\bar{\nu}_\mu}
\newcommand{\ant}{\bar{\nu}_\tau}
\newcommand{\ans}{\bar{\nu}_s}

\newcommand{\nee}{\nu_e \to \nu_e}
\newcommand{\nem}{\nu_e \to \nu_\mu}
\newcommand{\net}{\nu_e \to \nu_\tau}
\newcommand{\nes}{\nu_e \to \nu_s}

\newcommand{\nme}{\nu_\mu \to \nu_e}
\newcommand{\nmm}{\nu_\mu \to \nu_\mu}
\newcommand{\nmt}{\nu_\mu \to \nu_\tau}
\newcommand{\nms}{\nu_\mu \to \nu_s}



\newcommand{\Pee}{P_{e  e}}
\newcommand{\Pem}{P_{e  \mu}}
\newcommand{\Pet}{P_{e  \tau}}
\newcommand{\Pes}{P_{e  s}}

\newcommand{\Pme}{P_{\mu  e}}
\newcommand{\Pmm}{P_{\mu\mu}}
\newcommand{\Pmt}{P_{\mu  \tau}}
\newcommand{\Pms}{P_{\mu  s}}


\newcommand{\Pte}{P_{P_{\tau e}}}
\newcommand{\Ptm}{P_{\tau  \mu}}
\newcommand{\Ptt}{P_{\tau  \tau}}
\newcommand{\Pts}{P_{\mu  s}}

\newcommand{\Paeae}{P_{\bar{e}  \bar{e}}}
\newcommand{\Paeam}{P_{\bar{e}  \bar{\mu}}}
\newcommand{\Paeat}{P_{\bar{e}  \bar{\tau}}}
\newcommand{\Paeas}{P_{\bar{e}  \bar{s}}}

\newcommand{\Pamae}{P_{\bar{\mu}  \bar{e}}}
\newcommand{\Pamam}{P_{\bar{\mu}  \bar{\mu}}}
\newcommand{\Pamat}{P_{\bar{\mu}  \bar{\tau}}}
\newcommand{\Pamas}{P_{\bar{\mu}  \bar{s}}}


\newcommand{\Patae}{P_{\bar{\tau}  \bar{e}}}
\newcommand{\Patam}{P_{\bar{\tau}  \bar{\mu}}}
\newcommand{\Patat}{P_{\bar{\tau}  \bar{\tau}}}
\newcommand{\Patas}{P_{\bar{\mu}  \bar{s}}}

\renewcommand{\th}[1][]{%
  \theta\ifx\\#1\\\else_\text{#1}\fi
}
\newcommand{\thm}[1][]{%
  \theta^\text{M}\ifx\\#1\\\else_\text{#1}\fi
}
\renewcommand{\t}[1]{\text{{#1}}}
\newcommand{\avg}[1]{\left\langle {#1} \right \rangle}
\newcommand*{\dm}[1][]{%
  \Delta m^2\ifx\\#1\\\else_\text{#1}\fi
}
\newcommand{\zreco}{\cos{(\theta_z^{reco})}}
\newcommand{\ztrue}{\cos{(\theta_z^{true})}}
\newcommand{\z}{\cos{(\theta_z)}}
\newcommand{\Ereco}{E^{reco}}
\newcommand{\Etrue}{E^{true}}
\newcommand{\Aeff}{A^\text{eff}}
\newcommand{\emm}{\epsilon_{\mu\mu}}
\newcommand{\emt}{\epsilon_{\mu\tau}}
\newcommand{\eet}{\epsilon_{e\tau}}
\newcommand{\eem}{\epsilon_{e\mu}}
\newcommand{\ett}{\epsilon_{\tau\tau}}
\newcommand{\ep}{\epsilon^\prime}

% \begin{document}
\section{DeepCore}\label{ch:DCmethod}
In this part, we use the publically available DeepCore data sample~\cite{DC2019data} which is an updated version of what was used by the 
IceCube collaboration in a $\nu_\mu$ disapprearance analysis~\cite{DC2018mudisappearance}.

The detector systematics include ice absorption and scattering, and overall, lateral, and head-on optical efficiencies of the DOMs. 
They are applied as correction factors using the best-fit points from the DeepCore 2019 $\nu_\tau$ appearance 
analysis~\cite{DC2019tauappearance}.

The data include 14901 track-like events and 26001 cascade-like events, both divided into eight 
$ \log_{10}E^{reco} \in [0.75,1.75]$ bins, and eight $\zreco \in [-1,1]$ bins. Each event has a Monte Carlo weight $w_{ijk,\beta}$,
from which we can construct the event count as
\begin{align}\label{eq:MCevents}
    N_{ijk} &= C_{ijk}\sum_{\beta}w_{ijk,\beta} \phi_\beta^\text{det}\,,
\end{align}
where $C_{k\beta}$ is the correction factor from the detector systematic uncertainty and $\phi_\beta^\text{det}$ is defined as Eq.~\ref{eq:propFlux}. We have now substituted the effect of the Gaussian smearing 
by treating the reconstructed and true quantities as a migration matrix. 

The oscillation parameters used on our DeepCore simulations are from the
best-fit in the global analysis in~\cite{nufit}: $\theta_{12} = \SI{33.44}{\degree},\, \theta_{13} = \SI{8.57}{\degree},\, \Delta m^2_{21} =  \SI{7.42}{\electronvolt^2}$, and we 
marginalize over $\dm$ and $\theta_{23}$.

We plot the event pull $(N_{NSI} - N_{SI})/\sqrt{N_{SI}}$ where $N_{(N)SI}$ are the numbers of expected events
assuming (non-)standard interactions in Fig.~\ref{fig:event_pulls}. This gives the normalized difference in the
number of expected events at the detector, and illustrates the expected sensitivity of DeepCore for the NSI parameters.
\section{PINGU}
The methodology behind the PINGU simulations are the same as with our DeepCore study~. We use the public MC~\cite{PINGUdata}, which allows us to construct the event count as in Eq.~\ref{eq:MCevents}.
However, since no detector systematics is yet modelled for PINGU, the correction factors $C_{ijk}$ are all unity.
As with the DeepCore data, the PINGU Monte Carlo is divided into eight 
$\log_{10}E^{reco} \in [0.75,1.75]$ bins, and eight $\zreco \in [-1,1]$ bins for both track- and cascade-like events. 
We generate "data" by predicting the event rates at PINGU with the following best-fit parameters from~\cite{nufit}, except for the CP-violating phase which is set to zero for simplicity.

\begin{align}\label{eq:PINGUparams}
    &\Delta m^2_{21} =  \SI{7.42e-5}{\electronvolt^2},\hspace{0.5em} \dm =  \SI{2.517e-3}{\electronvolt^2}, \nonumber \\
    &\theta_{12} = \SI{33.44}{\degree},\hspace{1em} \theta_{13} = \SI{8.57}{\degree},\hspace{1em} \theta_{23} = \SI{49.2}{\degree}, \hspace{1em} \delta_\text{CP} = 0\,.
\end{align}
% \bibliographystyle{nature}
% \bibliography{ref.bib}
% \end{document}