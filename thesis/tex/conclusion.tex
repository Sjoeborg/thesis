\chapter{Conclusion}\label{ch:conc}
We were able to simulate the IceCube experiment with 86 strings, with our $3+0$ hypothesis shown against the IceCube Monte Carlo in Fig~\ref{fig:IC_MC_norm}. 
We normalized against the Monte Carlo, and produced the exclusion contour between the sterile neutrino parameters $\dm[41]$ and $\theta_[24]$ in Fig~\ref{fig:error_tuning}.
We found that the data is consistent with the $3+0$ hypothesis, with deviations up to 6\% at $\SI{10}{\TeV}$.
We also investigated the case when $\theta_{24} = \theta{34}$, and which produced an identical exclusion contour as in the case with $\theta_{34}=0$. This 
observation makes us conclude that IceCube is currently insensitive to any impact of $\theta_{34}$, with track events alone. However,
when opening up the $\tau$ channel by allowing $\theta_{34} \neq 0$, it is possible that with an inclusion of cascade events from $\nt$, IceCube could be able to observe 
an impact of $\theta_{34}$.


% TODO Conclusion:
% We produced the conclusion plots, and ic is insensitive to any impact if th34.
% NSI: which detector is more sensitive? Add paragraph on what parameters that PINGU can measure (in future tense). Write what constrain which parameters?
% Include effect on system uncertainty on PINGU. Compare with other measurements. Write from fig 5.7 that let Eem are anti-correlated, while no significant correlation can be seen from emt-emt, emt-eet.
% Reference papers on PINGU, NSI. And NSI from Llong baseline experiments. Include reference to DUNE from ESS paper with S+T+M.