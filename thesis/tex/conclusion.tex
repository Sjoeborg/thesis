\chapter{Conclusion}\label{ch:conc}
We were able to simulate the IceCube experiment with 86 strings, with our $3+0$ hypothesis shown against the IceCube Monte Carlo in Fig~\ref{fig:IC_MC_norm}. 
We normalized against the Monte Carlo, and produced the exclusion contour between the sterile neutrino parameters $\dm[41]$ and $\theta_[24]$ in Fig~\ref{fig:error_tuning}.
We found that the data is consistent with the $3+0$ hypothesis, with deviations up to 6\% at $\SI{10}{\TeV}$.
Our best-fit parameters were found to be$\dm[41] = \SI{0.01}{\eV^2}$ and $\theta_{24} = 0.67$ ($\sin^22\theta_{24} = 0.95$) at 
a p-value of $33\%$.
We also investigated the case when $\theta_{24} = \theta{34}$, and which produced an identical exclusion contour as in the case with $\theta_{34}=0$. This 
observation makes us conclude that IceCube is currently insensitive to any impact of $\theta_{34}$, with track events alone. However,
when opening up the $\tau$ channel by allowing $\theta_{34} \neq 0$, it is possible that with an inclusion of cascade events from $\nt$, IceCube could be able to observe 
an impact of $\theta_{34}$.

In our second part of the work, we put the sterile neutrino completely aside, and turned our focus to non-standard interactions. 
We included DeepCore in order to both lower the energy range of the data, but also to include cascade events in this range. 
Additionally, we simulated a proposed upgrade to DeepCore, PINGU, and produced data for it assuming a standard $3+0$ hypothesis with parameters from\cite{nufit}.
We saw that, on a probability level, in the high energy region of IceCube, NSI effects other than those stemming from $\emt$ are not apparent.
Moreover, we saw in the $\Pme$ shown in Fig~\ref{fig:eem_eet_prob} that we could expect a correlation between $\eem$ and $\eet$ in the single-digit \si{\GeV} range.
We also saw that $\emt$ effects were visible on probability level in both the \si{\GeV} and \si{\TeV} regions, making us hope to constrain $\emt$ further by 
a joint analysis.
In Fig~\ref{fig:event_pulls}, we saw that we can expect PINGU to be more sensitive to NSI effects than DeepCore, with $\emt$ effects being the weakest 
due to their low effect on all probability channels in the \si{\GeV} range.

% TODO Conclusion:
% NSI: which detector is more sensitive? Add paragraph on what parameters that PINGU can measure (in future tense). Write what constrain which parameters?
% Include effect on system uncertainty on PINGU. Compare with other measurements. Write from fig 5.7 that let Eem are anti-correlated, while no significant correlation can be seen from emt-emt, emt-eet.
% Reference papers on PINGU, NSI. And NSI from Llong baseline experiments. Include reference to DUNE from ESS paper with S+T+M.