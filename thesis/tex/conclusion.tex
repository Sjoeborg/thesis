\chapter{Conclusion}\label{ch:conc}
In this thesis, we have discussed the potential of explaining new physics at the detectors handled by the IceCube collaboration.
We started off by introducing the Standard Model in Sec.~\ref{ch:SM}, and noted that it has several shortcomings. We saw that, by adding a right-handed neutrino field,
neutrino masses are generated by the Higgs mechanism, thus enabling neutrino oscillations to occur. 
We mentioned the Super-Kamiokande experiment, and that their 1998 result found no other explanation than $\ne \to \nm$ mixing~\cite{sk1998}. In Sec.~\ref{ch:oscillation},
we saw how the neutrino mixing matrix could be constructed, and derived formulas describing neutrino oscillations in the two- and three-neutrino pictures. Adding a term for matter potential,
we then arrived at a Hamiltonian that we used in the time-dependent Schrödinger equation to solve for the oscillation probabilities in matter.

In Sec.\ref{sec:anomalies}, we learned that the LSND experiment found an excess in neutrino events that could be explained by a fourth, sterile neutrino~\cite{lsnd}.
We modified the Hamiltonian to incorporate mixing with this new neutrino species, and saw how $\ns$ affects the $\Pamam$ probability, giving us a resonant $\anm$ disappearance.
We then put the sterile neutrino aside, and introduced Non-Standard Interactions in Sec.\ref{sec:nsiTheory} as an possible example of new physics beyond the Standard Model.
We then studied the effects of NSI on the neutrino oscillation probabilities and saw that NSI effects other than those stemming from $\emt$ are not apparent in the \si{\TeV} region.
Moreover, we saw in the $\Pme$ that we could expect a correlation between $\eem$ and $\eet$ in the single-digit \si{\GeV} range.
We also saw that $\emt$ effects were visible on probability level in both the \si{\GeV} and \si{\TeV} regions.

In order to test the sterile neutrino hypothesis, we simulated the IceCube detector, configured with 86 strings. Our method was explained in Sec.\ref{ch:ICmethod}, and we only looked at track events. 
As a result, we were able to in Sec.~\ref{sec:sterileResults} confirm that the $3+0$ hypothesis holds by using a $\chi^2$ test between the data and our simulated event counts. %TODO: p value?
After simulating event counts for the $3+1$ hypothesis, which includes the sterile neutrino, we were able to find best-fit parameters of 
$\dm[41] = \SI{0.01}{\eV^2}$ and $\theta_{24} = 0.67$ ($\sin^22\theta_{24} = 0.95$) at 
a p-value of $33\%$. Even though we found a best-fit for two parameters of the sterile neutrino, they are not statistically significant. Hence, we found no evidence of a sterile neutrino in IceCube.
We also investigated the case when $\theta_{24} = \theta_{34}$, which produced an identical exclusion contour as in the case with $\theta_{34}=0$. From this observation, 
we concluded that IceCube is currently insensitive to any impact of $\theta_{34}$, with track events alone. However,
when opening up the $\tau$ channel by allowing $\theta_{34} \neq 0$, it is possible that with an inclusion of cascade events from $\nt$, IceCube could be able to better observe 
an impact of $\theta_{34}$. With this knowledge, we then completely left the sterile neutrino behind.

In Sec.~\ref{sec:constraining}, we moved on to the NSI parameter $\emt$ and its effect events captured by IceCube. 
Using a $\chi^2$ test, we found that the bound from IceCube on $\emt$ at 90 \% confidence level was 
\begin{align}
    -0.015 <\,& \emt < 0.014\ \quad (90\% \text{ CL})
\end{align}
for our `baseline' case of 15\% uncorrelated systematic uncertainty.

In Sec.~\ref{ch:DCmethod}, we moved on and explained our method of simulating a second detector, DeepCore. Armed with simulations possible for this detector, we then continued to explore the possibility of NSI, this time including both track and cascade events and a lower energy region.
Using the $\chi^2$ test defined in Eq.~\ref{eq:chisq_NSI} and marginalizing out $\dm[31]$ and $\theta_{23}$, we found the following constraints on the NSI parameters at 90\% confidence level:
\begin{align}
    -0.054 <&\, \ett < 0.067 \nonumber \\
    -0.029 <&\, \emt < 0.0070 \nonumber \\
    -0.12 <&\, \eem < 0.15 \nonumber \\
    -0.084 <&\, \eet < 0.15\ \quad (\text{all at }90\% \text{ CL})\,.
 \end{align}
We compared these constraints to literature, and found that the bounds for $\eem$ and $\eet$ were more stringent than those obtained in~\cite{demidov}.
Then, we did a joint $\chi^2$ test using simulated events from both IceCube and DeepCore. At 90\% confidence level, the bound of $\emt$ improved to
\begin{align}
    -0.016 <&\, \emt < 0.0070\ \quad (90\% \text{ CL})\,.
 \end{align}

In Sec.~\ref{ch:PINGUmethod} we described our method of simulating a proposed upgrade to the IceCube array: PINGU.
Since PINGU is not yet live, we produced data for it assuming a standard $3+0$ hypothesis with parameters from~\cite{nufit}.
After a $\chi^2$ analysis, we obtained the following constrains on the NSI parameters at 90\% confidence level:
\begin{align}
    -0.049 <&\, \ett < 0.057 \nonumber \\
    -0.010 <&\, \emt < 0.011 \nonumber \\
    -0.137 <&\, \eem < 0.11 \nonumber \\
    -0.132 <&\, \eet < 0.125 \quad (\text{all at }90\% \text{ CL})\,.
 \end{align}
After that, we did a joint $\chi^2$ analysis combining IceCube, DeepCore and simulated PINGU events to further constrict $\emt$. The bound at 90\% confidence level obtained was
\begin{align}
    -0.010 <&\, \emt < 0.0060\,. \quad (90\% \text{ CL})
 \end{align}

We were then able to draw the conclusion that we we expect PINGU to successfully be able to observe the impact of NSI, and a joint analysis between PINGU and at least one of the other detectors under the IceCube collaboration
will be able to even further constrain the parameters. We can expect PINGU to be more sensitive to NSI effects than DeepCore. Moreover, we saw that when constraining $\eem$ and $\eet$ from the negative side, 
we can expect the result to be highly dependent on the systematic uncertainty of PINGU.

Finally, we simulated the event count in PINGU allowing $\emt$, $\eem$, and $\eet$ to vary, setting $\ett=0$.
After and marginalizing out $\dm[31]$, $\theta_{23}$ and$\emt$, the effect of $\eem$ and $\eet$ we saw on $\Pme$ had indeed propagated through to the event count,
manifesting as an anti-correlation between the $\eem$ and $\eet$. No other pair of NSI parameters were observed to have 
a significant correlation.