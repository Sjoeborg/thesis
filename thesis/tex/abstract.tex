\chapter*{Abstract}
\addcontentsline{toc}{section}{Abstract}

The failure of the Standard Model of particle physics to predict neutrino masses invites us to amend it. We have no reason to believe that 
the interactions currently described by the gauge group are the whole picture, nor should we expect that the number of fermions hitherto observed is correct.
In this thesis, we explore two amendments to the Standard Model, and examine whether either of these are consistent with measured data.

Firstly, we consider the sterile neutrino, which has no interactions described by the Standard Model. We examine the effect of this 
new fermion on the neutrino oscillation probabilities, and present what could be a detectable signal in the \si{\TeV} energy range.
Secondly, we consider interactions beyond the Standard Model, possibly stemming from a higher-order theory.
We show how the parameters of these Non-Standard Interactions (NSI) can modify the oscillation probabilities and within which energy range we expect to discern this signal.
We use data from and simulate event counts in two Cherenkov detectors: IceCube and DeepCore. Moreover, we generate data and simulate a proposed upgrade of the DeepCore detector: PINGU. 

Using IceCube track events, we obtain best-fit values $\dm[41] = \SI{0.01}{\eV^2}$ and $\theta_{24} = 0.67$ for our sterile neutrino hypothesis at 
a p-value of $20\%$, which is not statistically significant. Hence, we found no evidence of a sterile neutrino in IceCube data. 
Moreover, we were unable to distinguish a signal from $\theta_{34}$ in our IceCube simulation.
We obtain stringent bounds on the NSI parameters and compare those to previous results in literature. We show that PINGU is expected to narrow the bound further 
on $\emt$, especially by considering a joint analysis with IceCube and DeepCore. Finally, we see that an anti-correlation between 
$\eem$ and $\eet$ at probability level was propagated down to event level, which we expect to be observable by PINGU.


\chapter*{Sammanfattning}
\addcontentsline{toc}{section}{Sammanfattning}
Misslyckandet med partikelfysikens s.k. standardmodell att förutsäga neutrinomassor uppmanar oss att förbättra den. Vi har ingen anledning att tro att
de interaktioner som för närvarande beskrivs av standardmodellens symmetrigrupp är hela bilden, och inte heller vi bör förvänta oss att antalet fermioner som hittils observerats
är korrekt.
I detta examensarbete undersöker vi två förlängningar av standardmodellen och undersöker om dessa är förenliga med uppmätta data.
Först introducerar vi den sterila neutrinon, som inte har några interaktioner som beskrivs av standardmodellen. Vi undersöker effekten av denna
nya fermion på neutrinoscillationssannolikheterna, och presentera vad som skulle vara en detekterbar signal i \si{\TeV}-området.
Vidare introducerar vi en ny interaktion som vi kallar Icke-Standard Interaktioner (NSI), som möjligen härrör från en högre ordningens teori.
Vi visar hur NSI-parametrarna kan modifiera oscillationssannolikheterna, och inom vilka energiområden vi förväntar oss att urskilja denna signal.
Vi simulerar två Cherenkov-detektorer: IceCube och DeepCore. Dessutom presenterar vi resultat som vi
kan förvänta oss att se i en föreslagen detektor: PINGU.

Genom att analysera spår från muoner i IceCube erhåller vi bästa anpassningsvärden $\dm[41] = \SI{0.01}{\eV^2}$ och $\theta_{24} = 0.67$ för vår sterila neutrinohypotes, med
ett p-värde på $20\%$, vilket inte är statistiskt signifikant. Därför hittade vi inga bevis på en steril neutrino i IceCube-datat.
Dessutom kunde vi inte urskilja en signal från $\theta_{34}$ i vår IceCube-simulering.
Vi uppmäter snäva gränser för NSI-parametrarna och jämför dem med tidigare resultat i litteraturen. Vi visar att PINGU förväntas minska gränsen ytterligare
på $\emt$, särskilt ifall man utformar en gemensam analys med IceCube och DeepCore. Slutligen ser vi att en anti-korrelation mellan
$\eem$ och $\eet$ på sannolikhetsnivå spreds ner till händelsenivå, något som vi förväntar oss kan observeras av PINGU.


\chapter*{Acknowledgements}
\addcontentsline{toc}{section}{Acknowledgements}
First and foremost, I would like to thank my supervisor Prof. Sandhya Choubey for her extensive guidance and support throughout,
and for the time she dedicated to my project. 

Secondly, I'm eternally grateful to everyone who contributed to making me be the person who I am. 
My friends from KTH,
mainly Svante, Anders, and Erik for making the final years -- in which I got to know you -- the best. 
My best friends, Gilbert and Max, with whom I have shared countless of evenings filled with laughter and 
great conversations. 
My beloved girlfriend Moa, who I met at the start of this thesis. I have no idea if this would have been possible without you.

Finally, my parents Anders and Helena, and also Ingemar. Thank you for nurturing my curiosity and for your unwavering belief in me.

\vspace{9cm}
The computations were enabled by resources provided by the Swedish
National Infrastructure for Computing (SNIC) at HPC2N, Umeå University
and NSC, Linköping University. The resources were partially funded by the Swedish Research 
Council through grant agreement no. 2018-05973