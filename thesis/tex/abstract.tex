\chapter*{Abstract}
\addcontentsline{toc}{section}{Abstract}

The failure of the Standard Model of particle physics to predict neutrino masses forces us to amend it. We have no reason to believe that 
the interactions currently described by the gauge group are the whole picture, nor should we expect that the number of fermions hitherto observed is correct.
In this thesis, we explore two modifications to the Standard Model, and examine whether these amendments are consistent with measured data.
Firstly, we consider the sterile neutrino, which have no interactions described by the Standard Model. We examine the effect of this 
new fermion on the neutrino oscillation probabilities, and present what would be a detectable signal in the \si{\TeV} range.
Secondly, we consider interactions beyond the Standard Model, possibly stemming from a higher-order theory.
We show how the parameters of these Non-Standard Interactions (NSI) can modify the oscillation probabilities, and within which energy range we expect to be able to discern this signal.
We use data from and simulate event counts in two Cherenkov detectors: IceCube and DeepCore. Moreover, we generate data and simulate a proposed upgrade of the DeepCore detector: PINGU. 
Using IceCube track events, we obtain best-fit values $\dm[41] = \SI{0.01}{\eV^2}$ and $\theta_{24} = 0.67$ for our sterile neutrino hypothesis at 
a p-value of $20\%$, which is not statistically significant. Hence, we found no evidence of a sterile neutrino in IceCube data. 
Moreover, were unable to distinguish a signal from $\theta_{34}$ in our IceCube simulation.
We obtain stringent bounds on the NSI parameters, and compare those to previous results. We show that PINGU is expected to further narrow the bound 
on $\emt$, especially the considering a joint analysis together with IceCube and DeepCore. Finally, we see that an anti-correlation between 
$\eem$ and $\eet$ at probability level was propagated down to event level, which we expect to be observable by PINGU.


\chapter*{Sammanfattning}
\addcontentsline{toc}{section}{Sammanfattning}
Misslyckandet med partikelfysikens s.k. standardmodell att förutsäga neutrinomassor tvingar oss att ändra den. Vi har ingen anledning att tro att
de interaktioner som för närvarande beskrivs av standardmodellens symmetrigrupp är hela bilden, och vi bör inte heller förvänta oss att antalet fermioner som hittils observerats.
I detta examensarbete undersöker vi två modifieringar av standardmodellen och undersöker om dessa ändringar är förenliga med uppmätta data.
Först introducerar vi den sterila neutrinon, som inte har några interaktioner som beskrivs av standardmodellen. Vi undersöker effekten av denna
nya fermion på neutrinoscillationssannolikheterna, och presentera vad som skulle vara en detekterbar signal i \si{\TeV}-området.
Vidare introducerar vi en ny interaktion som vi kallar Icke-Standard Interaktioner (NSI), som möjligen härrör från en högre ordningens teori.
Vi visar hur NSI-parametrarna kan modifiera oscillationssannolikheterna, och inom vilka energiområden vi förväntar oss att urskilja denna signal.

Vi samlar in data från och simulerar två Cherenkov-detektorer: IceCube och DeepCore. Dessutom presenterar vi resultat som vi
kan förvänta sig att se i en kommande uppgradering av DeepCore-detektorn: PINGU.

Vi får resultat som indikerar att den aktuella 3+0-hypotesen överensstämmer med IceCube-data, med bästa anpassningsvärden
$\dm[41] = \SI{0.01}{\eV^2}$ och $\theta_{24} = 0.67 $ med
ett p-värde på $33\%$. Vi kunde inte skilja en signal från $\theta_{34}$.
Vi erhåller stringenta gränser för NSI-parametrarna och jämför dem med tidigare resultat. Vi visar att PINGU förväntas minska gränsen ytterligare
hos $\emt$, särskilt genom att gemensamt analysera resultaten med IceCube och DeepCore. Slutligen ser vi att en anti-korrelation mellan
$\eem$ och $\eet$ på sannolikhetsnivå spreds ner till händelsenivå, vilket kommer att kunna observeras av PINGU.

\chapter*{Acknowledgements}
\addcontentsline{toc}{section}{Acknowledgements}
The computations were enabled by resources provided by the Swedish
National Infrastructure for Computing (SNIC) at HPC2N, Umeå University
and NSC, Linköping University. The resources were partially funded by the Swedish Research 
Council through grant agreement no. 2018-05973