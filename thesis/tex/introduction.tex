\chapter{Introduction}
There is no reason for us to believe that our current knowledge in particle physics is complete.
The most successful framework -- the Standard Model -- has notable shortcomings. Neither does it 
explain why there is something rather than nothing, nor does it give a candidate for dark matter\footnote{In the Standard Model,
matter and antimatter come in pairs of one particle and one antiparticle. When interacting,
antimatter \emph{annihilates} the matter, resulting in pure light. Since the universe exists,
there must have been more matter than antimatter at the Big Bang. %TODO: check and make better
Dark matter is heavy matter which does not interact with electromagnetic radiation.
No such particle exists in the Standard Model.}.

To make matters even worse, three of the massive particles are massless in the Standard Model.
In this thesis, we focus on the said particles -- the neutrinos -- and propose extensions to the Standard Model related to them. 
We then compare the effects of these extensions with collected data to see if our new theory more closely describes Nature or not. 

Ultimately, the purpose of this excursion is to guide future research where to uncover more accurate theories.
Traces of this grander theory will be present as breadcrumbs scattered in Nature. We just need to know where and how carefully to look. 

\section{Outline}
This thesis is outlined as follows. In Chapter~\ref{ch:osc}, we briefly review the most relevant parts of the Standard Model that will be relevant for our extensions of it.
Then we present the evidence and subsequent discovery of neutrino oscillations, which ultimately led to the 2015 Nobel physics prize, along with the modifications to the Standard Model needed to accommodate the discovery.
The reader is then introduced to a standard but detailed derivation of the neutrino mixing matrix and oscillation Hamiltonian, including the matter effects stemming from charged and neutral current weak interactions within the Earth.

In Chapter~\ref{ch:ic}, we present an experimental procedure used to observe signals from neutrinos originating from cosmic ray interactions in the atmosphere of the Earth. 
Furthermore, present the two present Antarctic neutrino detectors: IceCube and DeepCore.
We also summarize a proposed detector upgrade: Precision IceCube Next Generation Upgrade (PINGU), which we will use as a forecast in our analysis.

Chapter~\ref{ch:theory} consists of two parts. In Section~\ref{sec:anomalies}, we introduce a new particle: the sterile neutrino. 
We present how neutrino oscillations are modified by this hypothesized fourth neutrino and how these modifications would appear in IceCube if the particle is present in Nature. 
In Section~\ref{sec:nsiTheory}, we completely set the sterile neutrino aside and instead turn to the interactions between neutrinos and the Earth. 
We again amend the Standard Model by considering exotic interactions from a higher energy theory, manifesting themselves as sub-leading modifications to the matter potential. 
Since these new interactions are not present in the Standard Model, we call them non-standard interactions (NSI). 

Finally, Chapter~\ref{ch:results} contains the result from our two separate extensions of the Standard Model.
In Section~\ref{sec:sterileResults} we present our findings for the sterile neutrino and compare those with literature. 
In Section~\ref{sec:nsiResults}, we present new constrained bounds on the NSI parameters by combining our results for IceCube, DeepCore, and PINGU.