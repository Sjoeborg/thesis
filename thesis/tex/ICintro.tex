% \documentclass{thesis}
% \usepackage[margin=0.5in]{geometry}
% \usepackage{graphicx, slashed, siunitx, subcaption}
% \usepackage[utf8]{inputenc}
% \usepackage{xr-hyper}
% \usepackage{physics,amsmath, amsfonts, siunitx, amssymb}
\usepackage[utf8]{inputenc}
\newcommand{\n}[1]{\ensuremath{\nu_{#1}}}
\newcommand{\na}{\ensuremath{\nu_\alpha}}
\newcommand{\nb}{\ensuremath{\nu_\beta}}
\newcommand{\ana}{\ensuremath{\bar{\nu}_\alpha}}
\newcommand{\an}[1]{\ensuremath{\bar{\nu}_{\text{#1}}}}
\newcommand{\anb}{\ensuremath{\bar{\nu}_\beta}}
\renewcommand{\a}{\ensuremath{\alpha}}
\renewcommand{\b}{\ensuremath{\beta}}
\newcommand{\ab}{\ensuremath{\alpha\beta}}

\renewcommand{\ne}{\ensuremath{\nu_e}}
\newcommand{\ns}{\ensuremath{\nu_s}}

\newcommand{\nee}{\ensuremath{\nu_e \to \nu_e}}
\newcommand{\nem}{\ensuremath{\nu_e \to \nu_\mu}}
\newcommand{\net}{\ensuremath{\nu_e \to \nu_\tau}}
\newcommand{\nes}{\ensuremath{\nu_e \to \nu_s}}

\newcommand{\nme}{\ensuremath{\nu_\mu \to \nu_e}}
\newcommand{\nmm}{\ensuremath{\nu_\mu \to \nu_\mu}}
\newcommand{\nmt}{\ensuremath{\nu_\mu \to \nu_\tau}}
\newcommand{\nms}{\ensuremath{\nu_\mu \to \nu_s}}


\newcommand{\nte}{\ensuremath{\nu_\tau \to \nu_e}}
\newcommand{\ntm}{\ensuremath{\nu_\tau \to \nu_\mu}}
\newcommand{\ntt}{\ensuremath{\nu_\tau \to \nu_\tau}}
\newcommand{\nts}{\ensuremath{\nu_\mu \to \nu_s}}

\newcommand{\Pee}{\ensuremath{P_{e  e}}}
\newcommand{\Pem}{\ensuremath{P_{e  \mu}}}
\newcommand{\Pet}{\ensuremath{P_{e  \tau}}}
\newcommand{\Pes}{\ensuremath{P_{e  s}}}

\newcommand{\Pme}{\ensuremath{P_{\mu  e}}}
\newcommand{\Pmm}{\ensuremath{P_{\mu\mu}}}
\newcommand{\Pmt}{\ensuremath{P_{\mu  \tau}}}
\newcommand{\Pms}{\ensuremath{P_{\mu  s}}}


\newcommand{\Pte}{\ensuremath{P_{P_{\tau e}}}}
\newcommand{\Ptm}{\ensuremath{P_{\tau  \mu}}}
\newcommand{\Ptt}{\ensuremath{P_{\tau  \tau}}}
\newcommand{\Pts}{\ensuremath{P_{\mu  s}}}

\newcommand{\Paeae}{\ensuremath{P_{\bar{e}  \bar{e}}}}
\newcommand{\Paeam}{\ensuremath{P_{\bar{e}  \bar{\mu}}}}
\newcommand{\Paeat}{\ensuremath{P_{\bar{e}  \bar{\tau}}}}
\newcommand{\Paeas}{\ensuremath{P_{\bar{e}  \bar{s}}}}

\newcommand{\Pamae}{\ensuremath{P_{\bar{\mu}  \bar{e}}}}
\newcommand{\Pamam}{\ensuremath{P_{\bar{\mu}  \bar{\mu}}}}
\newcommand{\Pamat}{\ensuremath{P_{\bar{\mu}  \bar{\tau}}}}
\newcommand{\Pamas}{\ensuremath{P_{\bar{\mu}  \bar{s}}}}


\newcommand{\Patae}{\ensuremath{P_{\bar{\tau}  \bar{e}}}}
\newcommand{\Patam}{\ensuremath{P_{\bar{\tau}  \bar{\mu}}}}
\newcommand{\Patat}{\ensuremath{P_{\bar{\tau}  \bar{\tau}}}}
\newcommand{\Patas}{\ensuremath{P_{\bar{\mu}  \bar{s}}}}

\renewcommand{\th}[1][]{%
  \theta\ifx\\#1\\\else_\text{#1}\fi
}
\newcommand{\thm}[1][]{%
  \theta^\text{M}\ifx\\#1\\\else_\text{#1}\fi
}
\renewcommand{\t}[1]{\ensuremath{\text{{#1}}}}
\newcommand{\avg}[1]{\ensuremath{\left\langle {#1} \right \rangle}}
\newcommand*{\dm}[1][]{%
  \Delta m^2\ifx\\#1\\\else_\text{#1}\fi
}
\newcommand{\zreco}{\ensuremath{\cos{(\theta_z^{reco})}}}
\newcommand{\ztrue}{\ensuremath{\cos{(\theta_z^{true})}}}
\newcommand{\emm}{\ensuremath{\epsilon_{\mu\mu}}}
\newcommand{\emt}{\ensuremath{\epsilon_{\mu\tau}}}
\newcommand{\eet}{\epsilon_{e\tau}}
\newcommand{\eem}{\epsilon_{e\mu}}
\newcommand{\ett}{\ensuremath{\epsilon_{\tau\tau}}}
\newcommand{\ep}{\ensuremath{\epsilon^\prime}}
\renewcommand{\ne}{\nu_e}
\newcommand{\nm}{\nu_\mu}
\newcommand{\nt}{\nu_\tau}
\newcommand{\ane}{\bar\nu_e}
\newcommand{\anm}{\bar\nu_\mu}
\newcommand{\ant}{\bar\nu_\tau}
% \begin{document}
\section{Neutrino detection}
We always observe neutrinos indirectly through their associated charged lepton. Regardless of the type of interaction (charged current via the $W$ boson, or neutral current
via the $Z$), a charged lepton exits with altered properties. The lepton is then detected, and the properties of the neutrino involved in the 
interaction is then deduced. This deduction is obviously imperfect, and this introduces complexities that we will handle in Ch.~\ref{ch:ICmethod}. 

In this work, we only study the detectors handeled by the IceCube collaboration. They are of Cherenkov type, which means that they detect 
the secondary charged lepton by its emitted Cherenkov light, produced from its travel through the Antarctic ice. 
If the charged leptons interact heavily with the ice, they will travel a short distance and emit a localized flash of 
Cherenkov light. This event is referred to as a cascade. The neutral current interactions involves quarks, which recoils and procudes
showers of hadrons. Also, charged current $\ne$ interactions also produce cascades. A cascade event 
is shown in Fig.~\ref{fig:events_cascade}.
If the charged leptons don't interact as much in the ice, they penetrate a larger part of it, emitting light and tertiary particles
as they go. This event is referred to as a track, and are often due to muon charged current interactions. A track event 
is shown in Fig.~\ref{fig:events_track}.

\begin{figure}\label{fig:events}
    \begin{center}
        \begin{subfigure}{0.4\textwidth}
            \centering
            \includegraphics[clip, trim=0cm 0cm 0cm 30cm, width=1\textwidth]{figures/cascade_event.pdf}
            \caption{Event classified as cascade}  
            \label{fig:events_cascade}
          \end{subfigure}
        \begin{subfigure}{0.4\textwidth}
            \centering
            \includegraphics[clip, trim=0cm 0cm 0cm 30cm, width=1\textwidth]{figures/track_event.pdf}
            \caption{Event classified as track} 
            \label{fig:events_track}
        \end{subfigure}
        \caption{The two event types distinguished in the IceCube detector.}
    \end{center}
\end{figure}

To detect the Cherenkov light, 60 Digital Optical Modules (DOMs) are placed on a long string up to \SI{17}{\metre} apart. 86 of these strings are then lowered
into \SI{2.5}{\km} deep boreholes in the ice. The holes are then sealed by refreezing the ice, resulting in a total of 5160 DOMs in a volume of approximately \SI{1}{\km^3}~\cite{weaverThesis}.

The strings and DOMs are not spaced evenly, making some parts of the detector more sensitive to certain energy ranges than other.
8 strings packed more tightly than the other 78, making that part of the detector sensitive to neutrino energies down to single digit \si{\GeV}. Due to 
this part being situated deep within the ice, it is referred to DeepCore. DeepCore will be treated as a separate and independent detector from the rest, which
retains the name IceCube. A view of the current setup can be seen in Fig.~\ref{fig:array}. In this work, we consider DeepCore data between \SIrange{5.6}{56}{\GeV} and IceCube data in the range \SIrange{0.5}{10}{\TeV}.
\begin{figure}\label{fig:array}
    \centering
    \includegraphics[width=0.5\textwidth]{figures/icecube2.png}
    \caption{View of the full IceCube array}
\end{figure}

In 2017, the PINGU Letter of Intent was published~\cite{PINGUletter}. The `Precision IceCube Next Generation Upgrade' is an upgrade that will 
supplement DeepCore, i.e. boosing the capabilites of neutrino detection at the \si{\GeV} scale. As the PINGU upgrade is not yet financed nor built, we are
not able to use any data from it. However, the collaboration has released preliminary simulations which we will use to see how the upgrade might improve
IceCube and DeepCore bounds. The PINGU simulations have the same structure as the DeepCore data, so any analysis referring to DeepCore will
also apply to PINGU except where noted. However, we treat the PINGU detector as independent of the DeepCore experiment.

\begin{figure}[t]\label{fig:flux_aeff}
    \includegraphics[width=0.5\textwidth]{figures/flux.pdf}\hspace{-1cm}
    \includegraphics[width=0.5\textwidth]{figures/aeff.pdf}
    \vspace{-2.5cm}\caption{\emph{Left panel:} Interpolated South Pole atmospheric flux with data from~\cite{hondaArticle}.
    \emph{Right panel:} Interpolated IceCube effective area with data from~\cite{ICaeff}.}
\end{figure}

\subsection{Atmospheric Neutrino Flux}
Atmospheric neutrinos originates from cosmic rays composed of protons interacting with nuclei in the atmosphere.
These interactions ultimately produces pions, which decay as 
\begin{align}\label{eq:pion}
    &\pi^+ \to \mu^+ + \nm\,, \quad \pi^- \to \mu^- + \anm \nonumber \\
    &\pi^+ \to e^+ + \ne\,,\,\, \quad \pi^- \to e^- + \ane\,.
\end{align}
In the muonic decay channel, muons are emitted which will be detected by the IceCube detector. A part of the 
uncorrelated systematic error comes from this \emph{muon background}, i.e. events misclassified as muons from 
$\nm$ interactions rather than from pion decay. Moreover, the atmospheric flux is often associated with a large error.
In this work, we will use a flux normalization error of 24\%, and a zenith slope error of 4\%~\cite{hondapaper}.

The flux is provided in \cite{hondaData,hondaArticle}, and a selection is shown in Table~\ref{table:flux}
The flux data is binned in $\ztrue$. The fluxes are averaged over azimuthal direction and over solar minimum/maximum. 
The units of the fluxes are given as \si{\per\GeV \per\metre\squared \per\second \per\steradian} and are omitted
from the table for clarity. 
We note that the fluxes for $\nt$ and $\nu_{\bar{\tau}}$ are missing. Kaons can decay into neutral pions, which in turn 
can produce $\nt$, but this branching ratio is extremely small. Thus, we never have to use probabilities on the form 
$P_{\tau \beta}$, since we have no incoming atmospheric $\nu_\tau$ flux. 
Interpolating the data yields makes us capable of returning all four necessary fluxes for a given true energy and true zenith.
The result is shown in Fig.~\ref{fig:flux_aeff}.

\begin{table}[h]\label{table:flux}
    \begin{center}
        \begin{tabular}{lcccccc}
            \hline \hline
            $\Etrue$ [\si{\GeV}] &$\phi_\mu$ &$\phi_{\bar{\mu}}$ &$\phi_e$ &$\phi_{\bar{e}}$ & $\ztrue$\\\\
            \hline
            27825 &  \SI{6.06e-12}{} &  \SI{3.17e-12}{} &  \SI{1.56e-13}{} &  \SI{1.04e-13}{} &   [-0.2, -0.1] \\
            247707 &  \SI{5.94e-16}{} &  \SI{2.92e-16}{} &  \SI{1.36e-17}{} &  \SI{8.12e-18}{} &   [-0.7, -0.6] \\
                22 &  \SI{3.33e-02}{} &  \SI{2.78e-02}{} &  \SI{9.57e-03}{} & \SI{7.15e-03}{} &   [-0.3, -0.2] \\
            432876 &  \SI{5.19e-17}{} &  \SI{2.32e-17}{} &  \SI{1.46e-18}{} & \SI{9.83e-19}{} &   [-1.1, -1.0] \\
            64280 &  \SI{1.58e-13}{} &  \SI{8.10e-14}{} &  \SI{3.49e-15}{} &  \SI{2.21e-15}{} &   [-0.4, -0.3] \\
            \hline \hline
        \end{tabular}
    \end{center}
    \caption{A selection of processed atmospheric South Pole fluxes from~\cite{hondaData} by Honda et al.~\cite{hondaArticle}.}
\end{table}


\subsection{Event Reconstruction}
After an event has occurred, the IceCube algorithms process the data coming from the detector to \emph{reconstruct} the event. This means that, given the parameters recorded by the detector, what are their "true" values?
We are interested in two variables: the energy and the direction. Each event is tagged with a probable energy and zenith angle, called the recostructed parameters $\Ereco$ and $\zreco$, which are the parameters according to the DOMs.
The collaboration then uses numerous sophisticated methods to backtrack the reconstructed parameters to the true parameters. So a charged lepton hits the DOMs, and we ultimately end up with the associated neutrino's true and reconstructed energy and zenith angle. The reconstructed parameters are what we are using to analyze the data (because this is what the detector actually sees), while the true parameters are used in the 
determination of that neutrino's `actual' flux and cross-section (because this is what Nature sees).

How do we then translate between the reconstructed and true parameters? In this work, we are using two different methods, which are based on the form of data available to us. They will be outlined in Sec.\ref{ch:ICmethod} and Sec.~\ref{ch:DCmethod}.



% \bibliographystyle{nature}
% \bibliography{ref.bib}
% \end{document}