\documentclass[twocolumn]{article}
\usepackage{physics,amsmath, amsfonts, siunitx, amssymb}
\usepackage[utf8]{inputenc}
%\documentclass[a4paper,10pt,draft]{thesis}
\usepackage{physics,amsmath, amsfonts, siunitx, amssymb, graphicx, slashed,subcaption}
\usepackage[utf8]{inputenc}
\usepackage[margin=1in]{geometry}
\usepackage[hidelinks]{hyperref}
\usepackage{xr-hyper}
\newcommand{\n}[1]{\nu_{#1}}
\newcommand{\na}{\nu_\alpha}
\newcommand{\nb}{\nu_\beta}
\newcommand{\ana}{\bar{\nu}_\alpha}
\newcommand{\an}[1]{\bar{\nu}_{\text{#1}}}
\newcommand{\anb}{\bar{\nu}_\beta}
\renewcommand{\a}{\alpha}
\renewcommand{\b}{\beta}
\newcommand{\ab}{\alpha\beta}


\renewcommand{\ne}{\nu_e}
\newcommand{\nm}{\nu_\mu}
\newcommand{\nt}{\nu_\tau}
\newcommand{\ns}{\nu_s}

\newcommand{\ane}{\bar{\nu}_e}
\newcommand{\anm}{\bar{\nu}_\mu}
\newcommand{\ant}{\bar{\nu}_\tau}
\newcommand{\ans}{\bar{\nu}_s}

\newcommand{\nee}{\nu_e \to \nu_e}
\newcommand{\nem}{\nu_e \to \nu_\mu}
\newcommand{\net}{\nu_e \to \nu_\tau}
\newcommand{\nes}{\nu_e \to \nu_s}

\newcommand{\nme}{\nu_\mu \to \nu_e}
\newcommand{\nmm}{\nu_\mu \to \nu_\mu}
\newcommand{\nmt}{\nu_\mu \to \nu_\tau}
\newcommand{\nms}{\nu_\mu \to \nu_s}



\newcommand{\Pee}{P_{e  e}}
\newcommand{\Pem}{P_{e  \mu}}
\newcommand{\Pet}{P_{e  \tau}}
\newcommand{\Pes}{P_{e  s}}

\newcommand{\Pme}{P_{\mu  e}}
\newcommand{\Pmm}{P_{\mu\mu}}
\newcommand{\Pmt}{P_{\mu  \tau}}
\newcommand{\Pms}{P_{\mu  s}}


\newcommand{\Pte}{P_{P_{\tau e}}}
\newcommand{\Ptm}{P_{\tau  \mu}}
\newcommand{\Ptt}{P_{\tau  \tau}}
\newcommand{\Pts}{P_{\mu  s}}

\newcommand{\Paeae}{P_{\bar{e}  \bar{e}}}
\newcommand{\Paeam}{P_{\bar{e}  \bar{\mu}}}
\newcommand{\Paeat}{P_{\bar{e}  \bar{\tau}}}
\newcommand{\Paeas}{P_{\bar{e}  \bar{s}}}

\newcommand{\Pamae}{P_{\bar{\mu}  \bar{e}}}
\newcommand{\Pamam}{P_{\bar{\mu}  \bar{\mu}}}
\newcommand{\Pamat}{P_{\bar{\mu}  \bar{\tau}}}
\newcommand{\Pamas}{P_{\bar{\mu}  \bar{s}}}


\newcommand{\Patae}{P_{\bar{\tau}  \bar{e}}}
\newcommand{\Patam}{P_{\bar{\tau}  \bar{\mu}}}
\newcommand{\Patat}{P_{\bar{\tau}  \bar{\tau}}}
\newcommand{\Patas}{P_{\bar{\mu}  \bar{s}}}

\renewcommand{\th}[1][]{%
  \theta\ifx\\#1\\\else_\text{#1}\fi
}
\newcommand{\thm}[1][]{%
  \theta^\text{M}\ifx\\#1\\\else_\text{#1}\fi
}
\renewcommand{\t}[1]{\text{{#1}}}
\newcommand{\avg}[1]{\left\langle {#1} \right \rangle}
\newcommand*{\dm}[1][]{%
  \Delta m^2\ifx\\#1\\\else_\text{#1}\fi
}
\newcommand{\zreco}{\cos{(\theta_z^{reco})}}
\newcommand{\ztrue}{\cos{(\theta_z^{true})}}
\newcommand{\z}{\cos{(\theta_z)}}
\newcommand{\Ereco}{E^{reco}}
\newcommand{\Etrue}{E^{true}}
\newcommand{\Aeff}{A^\text{eff}}
\newcommand{\emm}{\epsilon_{\mu\mu}}
\newcommand{\emt}{\epsilon_{\mu\tau}}
\newcommand{\eet}{\epsilon_{e\tau}}
\newcommand{\eem}{\epsilon_{e\mu}}
\newcommand{\ett}{\epsilon_{\tau\tau}}
\newcommand{\ep}{\epsilon^\prime}





\begin{document}

\section{Notes from~\cite{IC2012}}.

DeepCorelowers the IC neutrino energy threshold to 10 GeV.
DC sits 2100 m below the ice, beneath the IC array.
It uses the existing IC array as a "veto" against the background downward cosmic muons. Together with increased module density, better PMTs, clear ice, and this "veto", DC improves the sensitivity.

There is a dust layer between 2000 and 2100 m that DC tries to avoid. There are no DOMs there because of the significant scattering of light. Instead, the remaining DOMs are placed directly above this region in a higher density.

\section{Notes from~\cite{giunti}}
\subsection*{5.1 Weak interactions}
Leptonic charged current
\begin{align}
  j^\rho_{Z,\nu} = 2\sum_{\alpha = e,\mu,\tau} \bar{\nu}_{\alpha L} \gamma^\rho \ell_{\alpha L}
= \sum_{\alpha = e,\mu,\tau} \bar{\nu}_\alpha \gamma^\rho(1-\gamma^5)\ell_\alpha
\end{align}
Leptonic charged current weak interaction Lagrangian
\begin{align}
  \mathcal{L}^{(CC)}_{\mathrm{int,L}} = - \frac{g}{2\sqrt{2}}\left( j^\rho_{W,L}W_\rho + {j^\rho_{W,L}}^{\dagger} W^\dagger_\rho \right)
\end{align}

Neutrino neutral current
\begin{align}
  j^\rho_{Z,\nu} = \sum_{\alpha = e,\mu,\tau} \bar{\nu}_{\alpha L} \gamma^\rho \nu_{\alpha L}
= \frac{1}{2} \sum_{\alpha = e,\mu,\tau} \bar{\nu}_\alpha \gamma^\rho(1-\gamma^5)\na 
\end{align}
Neutrino neutral current weak interaction Lagrangian
\begin{align}
  \mathcal{L}_{\mathrm{I}, \nu}^{(\mathrm{NC})}=-\frac{g}{2 \cos \vartheta_{\mathrm{W}}} j_{Z, \nu}^{\rho} Z_{\rho}
\end{align}

To lowest perturbation order, neutrinos only interact with free leptons, whose interaction amplitudes can be calculated exactly.

\subsection*{5.3 Neutrino-nucleon scattering}
When considering nucleon scatterings, we need the parton distribution functions of the nucleon. We use the quark-parton model of hadrons. According to this model, a nucleon is a composite system of
three valence quarks and a sea of quark–antiquark pairs of all flavors.

\subsection*{6.1 Dirac masses}
Neutrino masses might be a low-energy manifestation of BSM physics. Their smallness might be due to a high-energy scale, perhaps for force unification (related to see-saw in some way).

Introduce right-handed components $\nu_{\alpha R}$ to the SM. The "Dirac neutrino" masses are then generated by the familiar Higgs mechanism, $m_k = y^\nu_k v / \sqrt{2}$. We then eliminate the asymmetry between the lepton and quark sectors. Since they are right-handed, the fields $\nu_{\alpha R}$ are singlets of both $SU(3)_C \times SU(2)_L$ and the electroweak $SU(2)_L \times U(1)_Y$, so they are invariant under the SM symmetries.


We can also construct sterile neutrinos from the left-handed chiral fields $\tilde{\nu}_{\alpha L} = \nu^C_{\alpha R}$. As long as they are singlets under the SM symmetries, they are per definition sterile. However, convention is calling all sterile fields right-handed. Chiral fermion fields are the building block of the SM.

The only characteristic that distinguishes the three charged
leptons is their mass and the flavor of a charged lepton is identified by measuring its
mass. Neutrinos are detected indirectly by observing the flavor of the lepton in weak interactions, so "flavor neutrinos are not required to have a definite mass", so instead they are superpositions of neutrinos in their mass eigenstates.

\subsection*{6.2 Majorana masses}
If $q \neq 0$, $\psi$ and $\psi^C$ obey different equations. Thus, $\psi \neq \psi^C$ and we don't have Majorana particles. Majorana spinors are simpler (only two components) than Dirac spinors, so Majorana neutrino's might be preferable due to Occhams razor. BSM neutrinos are usually Majorana. Dirac vs Majorana neutrinos can't be distinguished from neutrino oscillations, since the oscillations are kinematic phenomena. Dirac and Majorana are kinematic ally equivalent since they satisfy the same energy-momentum dispersion relation.  

It is important, however, to realize that the SM cannot be considered as
the final theory of everything, but only as an effective low-energy theory which is
the low-energy product of the symmetry breaking of a high-energy unified theory. The Lagrangian has energy dimension 4, so a coupling constant of a dimension $d$ operator is proportional to $\mathcal{M}^{4-d}$ of the high-energy theory. This dependence suppressed the low-energy effect of SM physics. A Majorana mass can be expressed in the Lagrangian as 
\begin{align*}
  \mathcal{L}_{5}=\frac{g}{\mathcal{M}}\left(L_{L}^{T} \tau_{2} \Phi\right) \mathcal{C}^{\dagger}\left(\Phi^{T} \tau_{2} L_{L}\right)+\mathrm{H.c.}
\end{align*}
which is proportional only to $\mathcal{M}^{-1}$. While not renormalizable in SM, the study of neutrino masses provides the most accessible low-energy window on BSM physics. A high-energy GUT can bleed into the low-energy SM through non-renormalizable Lagrangian terms, whose effects are suppressed by the coupling constants.

The Majorana mass term 
\begin{align*}
  \mathcal{L}_{\mathrm{mass}}^{\mathrm{M}}=\frac{1}{2} n_{L}^{T} \mathcal{C}^{\dagger} M \boldsymbol{n}_{L}+\mathrm{H.c.}=\frac{1}{2} \sum_{k=1}^{3} m_{k} \nu_{kL}^{T} \mathcal{C}^{\dagger} \n{k L}+\mathrm{H.c.}
\end{align*}
is not invariant under the global $U(1)$ gauge transformation
\begin{align*}
  \n{k L} \to e^{i \varphi} \n{k L}\,.
\end{align*}
This implies a violation of the total lepton number. Moreover, this asymmetry means that the Majorana mixing matrix has three physical CP-violating phases, apart from just one in the Dirac case. Thus, the $3 \times 3$ Majorana mixing matrix depends of three mixing angles and three physical CP-violating phases. 

These can be grouped as the Dirac phase (three mixing angles and one phase) and as the Majorana phase (two phases):
$U=U^D D^M$
$U^D$ has similarities to the quark mixing matrix CKM. $D^M$ is usually parametrized as 
\begin{align*}
  \text{diag}(1, e^{i \lambda_2},e^{i \lambda_3})
.\end{align*} 

If only $\nu_L$ exists, the neutrino Lagrangian only contains the Majorana mass term (which is not invariant under $SU(2)_L \times U(1)_Y$)
\begin{align*}
  \mathcal{L}^L_\text{mass} = \frac{1}{2} m_L \n{L}{T} \mathcal{C}^\dagger \n{L} + \text{H.c.}\,.
\end{align*}
If $\nu_R$ also exists, we can have both the Dirac and Majorana mass terms in the neutrino Lagrangian:
\begin{align*}
  \mathcal{L}^D_\text{mass} &= -m_D \an{R} \n{L} + \text{H.c.} \\
  \mathcal{L}^R_\text{mass} &= \frac{1}{2}m_R \n{R}{T} \mathcal{C}^\dagger \n{R} + \text{H.c.}\,.
\end{align*}
Together, we can then have the general Dirac-Majorana neutrino mass term 
\begin{align*}
  \mathcal{L}^{D+M} = \mathcal{L}^D + \mathcal{L}^L + \mathcal{L}^R\,,
\end{align*}
which through calculations implies that massive neutrinos are Majorana if we have a Dirac-Majorana mass term.

Conclusions:
\begin{itemize}
  \item With one generation of $\nu_L$ and $\nu_R$, there are two massive Majorana fields $\n{1}$ and $\n{2}$
  \item Convention: refer to $\nu_L$ and $\nu_R^C$ as the left-handed fields in the flavor basis.
  \item Mixing between $\nu_L$ and $\nu_R^C$ implies active-sterile neutrino oscillations.
\end{itemize}

\subsection{Weak interactions}
The active field $\nu_L$ and the sterile field $\nu_R$ are linear combinations of the same (massive) neutrino fields $\nu_{1L}$ and $\nu_{2L}$. CC interactions create superpositions of these, which we can see from the CC Langrangian
\begin{align*}
  \mathcal{L}^\text{CC}_\text{int} = -\frac{g}{\sqrt{2}} \sum_{k=1,2} (U^*_{1k}\bar{\nu}_{kL}\gamma^\mu \ell_L W_\mu + U_{1k}\bar{\ell}_L \gamma^\mu \nu_{kL} W^\dagger_\mu)
\end{align*}
Depending on their masses, we get a linear combination of active and sterile neutrinos, with a oscillatory probability of detecting the active part. The sterile part manifests itself through a survival probability of active neutrinos smaller than one.

For NC, we can have transitions among different massive neutrinos since we have the Lagrangian
\begin{align*}
  \mathcal{L}^\text{NC}_\text{int} = \frac{-g}{2\cos{\theta_W}} \sum_{k,j=1,2} U^*_{1k}U_{1j} \bar{\nu}_{kL}\gamma^\mu \nu_{jL} Z_\mu
\end{align*}

\subsection{CP invariance}
A real mass matrix implies CP invariance. The mixing matrix $U$ can contain imaginary elements while maintaining CP invariance. The column matrix of left-handed fields transforms as 
\begin{align*}
  N_L \overset{\text{CP}}{\to} \cup_\text{CP} N_L \cup_{\mathrm{CP}}^{-1} = i \gamma^0 C \bar{N}_L^T
\end{align*}
Likewise, the chiral left-handed fields $n_L$ $(N_L = Un_L)$ transform as
\begin{align*}
  n_L \overset{\text{CP}}{\to} \cup_\text{CP} n_L \cup_{\mathrm{CP}}^{-1} = i U^\dagger U^* \gamma^0 C \bar{n}_L^T = i (\rho \mathcal{O}^T \mathcal{O}\rho)^* \gamma^0 C \bar{n}_L^T = i \rho^2 \gamma^0 C \bar{n}_L^T\,,
\end{align*}
so the  CP parities of the massive Majorana fields are determined by the phases $\rho_k$ as
\begin{align*}
  \chi^\text{CP}_k = i \rho^2_k\,.
\end{align*}
Finally, the column matrix $n = n_L + n_L^C$ transforms as
\begin{align*}
  n \overset{\text{CP}}{\to}  i \rho^2 \gamma^0 n\,.
\end{align*}
So a Majorana neutrino has imaginary CP parity with sign determined by $\rho^2_k$
\begin{itemize}
  \item If $m_L = m_R = 0$, we have a Dirac field which is equivalent to two Majorana fields with the
same mass and opposite CP parities. A Dirac field has a definite CP parity $\chi^\text{CP} = i$.
  \item If $\abs{m_L}, m_R \ll m_D$, we have pseudo-Dirac neutrinos which are Majorana. 
  \item If $m_D \ll m_R, \quad m_L = 0$, we have the see-saw $m_1 \simeq \frac{m_D^2}{m_R}, m_2 \simeq m_R$, where the heavy mass of $\nu_2$ is responsible for the light mass of $\nu_1$. They have a small mixing angle, which implies that $\nu_1$ mainly consists of $\nu_L$ and $v_2$ mainly of $\nu_R$. 
\end{itemize}

The assumption $m_L = 0$ is "natural" since $m_L$ is forbidden in SM. $m_D$ can be generated by the Higgs, and is thus proportional to the symmetry-breaking scale (electroweak) \SI{e2}{GeV}. So the symmetries of SM "protects" $m_D$, while $m_R$ (being a singlet of the SM symmetries) is not. If $m_R$ is generated by BSM physics, $\nu_R$ can belong to a non-trivial multiplet of their symmetries. Then, $m_R$ is protected by the high-energy theory and can have a GUT scale (\SIrange{e14}{e16}{GeV}). Thus, the see-saw model produces the ratio $m_D/m_R \sim 10^{-14}-10^{-12}$. 

\subsection*{7 Neutrino oscillations}
Neutrinos in oscillation experiments are ultrarelativistic. Only neutrinos with energy lager than 100 keV can be detected.
The number of massive neutrinos are equal to of greater than three. If it is greater than three, the additional neutrinos in the flavor basis are sterile. 

Plane wave computations lead to the transition probability (using $t \approx L$ for astronomical distances)

\begin{align*}
  P_{\na \to \nu_\beta} 
  &= \sum_{k,j} U^*_{\alpha k} U_{\beta k} U_{\alpha j} U^*_{\beta j} \exp\left( -i \frac{\Delta m_{kj}^2 L}{2E}\right) \\
  &= \sum_k \abs{U_{\alpha k}}^2 \abs{U_{\beta k}^2} \\
  & \qquad + 2 \text{Re}\sum_{k>j} U^*_{\alpha k} U_{\beta k} U_{\alpha j} U^*_{\beta j} \exp \left( -2\pi i \frac{L}{L^\text{osc}_{kj}}\right) \\
  &= \delta_{\alpha \beta} - 4\sum_{k > j} \text{Re}[U^*_{\alpha k} U_{\beta k} U_{\alpha j} U^*_{\beta j}] \sin^2\left(\frac{\Delta_{kj}^2L}{4E}\right) \\
  &+ 2\sum_{k>j} \text{Im}[U^*_{\alpha k}U_{\beta k}U_{\alpha j} U^*_{\beta j}]\sin\left( \frac{\Delta m_{kj}^2 L}{2E} \right)
\,,\end{align*}
where we have defined the oscillation lengths $L^\text{osc}_{kj} = 4\pi E / \Delta m_{kj}^2$. For antineutrinos (Dirac antineutrinos with positive helicity or Majorana neutrinos with positive helicity), we have the same expression except for the sign in from of the imaginary part, and the quartic product $U_{\alpha k}U^*_{\beta k}U^*_{\alpha j} U_{\beta j}]$ instead.
Measurements of neutrino oscillations yield only information about the squared mass-differences $\Delta m_{kj}^2$ and the elements of the mixing matrix $U$. Moreover, we have quartic products of $U$. They are independent of the parametrization since they are invariant under the rephasing transformation 
\begin{align*}
  U_{\alpha k} \to e^{i \psi_\alpha} U_{\alpha k} e^{i\phi_k}
,\end{align*}
so the quartic products are independent of the factorized phases of the mixing matrix. This corresponds to a rephasing of the charged lepton and neutrino fields. If we have Majorana neutrinos, we have two Majorana phases in the diagonal matrix to the right of the mixing matrix. These phases are not included in the quartic product, and are thus unmeasurable in neutrino oscillation experiments. 

The oscillating term depends on the coherence of massive neutrino components. If different massive neutrinos are produced in an incoherent way, the transition probability $P$ reduces to the constant term. This also happens if we have an uncertainty in the distance $L$ or when we average over the energy resolution of the detector, since the average of the exponential is zero.

The oscillation probabilities of the channels with $\alpha \neq \beta$ are called transition probabilities, while the channels with $\alpha = \beta$ are called survival probabilities. For the survival probability of $\alpha$, we get 
\begin{align*}
  P_{\na \to \nu_\alpha} = 1-4\sum_{k > j}\abs{U_{\alpha k}}^2 \abs{U_{\alpha j}}^2 \sin^2 \left( \frac{\Delta m_{kj}^2 L}{4E} \right)
.\end{align*}

The average transition probability has a maximum at $\frac{1}{N}$, and the average survival probability a minimum at $\frac{1}{N}$. If all elements of the $N \times N$ matrix $U$ have the same absolute value, we have N-maximal mixing: minimal average survival probability equal to $\frac{1}{N}$ and maximal average transition probability equal to $\frac{1}{N}$ in each possible channel. 

Neutrinos mass states produced/detected in CC interactions are linearly combined by the mixing matrix $U$ to form flavor states. If the experiment is not sensitive to the differences of the contribution of the different neutrino masses, these flavor states reduce to this standard linear combination. 

\subsection*{7.3 Symmetries}
The neutrino oscillation theory is based on a local quantum field theory, and is thus CPT invariant. Thus,
\begin{align*}
  P_{\na \to \nb} &= P_{\anb \to \ana} \\
  P_{\na \to \na} &= P_{\ana \to \ana} 
.\end{align*}
So the survival probabilities for neutrinos and antineutrinos are equal. If the description of nature through local quantum field theories are only approximate, we could expect the CPT invariance to be violated. 
regarding CP invariance, we know that this is violated due to the complex elements of the three-neutrino mixing matrix. Thus, T symmetry should also be violated in order to keep CPT invariant.

We can inspect these violations by studying the asymmetry parameters
\begin{align*}
  A^\text{CPT}_{\alpha \beta}     &= P_{\na \to \nb} - P_{\anb \to \ana} \\
  A^\text{CP}_{\alpha \beta}       &= P_{\na \to \nb} - P_{\ana \to \anb} \\
  A^\text{T}_{\alpha \beta}       &= P_{\na \to \nb} - P_{\nb \to \na}   \\
  \bar{A}^\text{T}_{\alpha \beta} &= P_{\ana \to \anb} - P_{\anb \to \ana}
.\end{align*}
The CPT symmetry implies that the CP asymmetry is antisymmetric, i.e.
\begin{align*}
  A^\text{CP}_{\alpha \beta} = - A^\text{CP}_{\beta \alpha}
,\end{align*}
which implies $P_{\ana \to \anb} = P_{\nb \to \na}$. Thus, a CP asymmetry can only be measured in transitions between different flavors. Moreover $A^\text{CP}$ only contains the imaginary oscillatory part. Thus, experiments must be sensitive to the oscillatory behavior of the transition probabilities, i.e. $\dm_{kj} / 2E$ needs to be of order one. If the phases are smaller, the probabilities are too small to be measured. If they are much greater than unity, the CP asymmetry shadows the sine.

Regarding the T violation, the CPT relation implies that 
\begin{align*}
  A^\text{T}_{\ab} = -\bar{A}^\text{T}_{\ab} = A^\text{CP}_{\ab}
.\end{align*}
Thus, measuring a CP asymmetry is equivalent to measuring a T asymmetry. We can then measure a T asymmetry by designing an experiment that is sensitive to the oscillations in the flavor transition probabilities (so we can see the CP asymmetry).

\subsection*{Two-neutrino mixing}
By restricting the mixing angle $0 \le \theta \le \pi / 2$, and since the mixing angle dependence of the transition probability is expressed by $\sin^2 2 \theta$ which is symmetric under the exchange $ \theta \to \pi / 2 - \theta $, there is a degeneracy of the transition probability for $\theta$ and $\pi / 2 - \theta$. The difference is that mixing angles $\theta < \pi / 4$ correspond to the case then the incoming neutrino $\na$ consists of more $\n{1}$ than $\n{2}$. When $\theta > \pi / 4$, $\na$ consists of more $\n{2}$ than $\n{1}$. This is not a feature in matter oscillations.

If 
\begin{align*}
  \frac{\dm L}{2E} \ll 1
\,,\end{align*}
the different flavors can't be measured.
If 
\begin{align*}
  \frac{\dm L}{2E} \gg 1
\,,\end{align*}
only the average transition probability is observable. We can then define the sensitivity as the value of $\dm$ for which 
\begin{align*}
  \frac{\dm L}{2E} \sim 1
,\end{align*}
so the sensitivity is then E[MeV] / L[m] = E[GeV]/L[km]
\subsection*{Experiment}
\begin{itemize}
  \item Appearance: Measure transitions between flavors. If the final flavor is not present in the incoming beam, the experiment is sensitive to small mixing angles.
  \item Disappearance: Measures the survival probability by counting the number of interactions with the expected one. Even we have no oscillations, we still have statistical fluctuations. Therefore, it is harder to measure small mixing angles here.
\end{itemize}

Since we have spacial uncertainties in both $L$ and $E$, it is always necessary to average the oscillation probability over the distributions of $L$ and $E$. For two-neutrino mixing, we have the following transition probability for $x = L / E$:
\begin{align*}
  P_{\na \to \nb}(x) = \frac{1}{2} \sin^2 2\theta \left[1 - \left\langle \cos \left( \frac{\Delta m^2 x}{2} \right) \right\rangle \right]
\,,\end{align*}
with
\begin{align*}
  \left \langle \cos \left (\frac{\dm}{2} x \right) \right \rangle = \int \cos \left (\frac{\dm}{2} x \right ) \, \phi(x) \, \dd x
.\end{align*}
Assuming Gaussian distribution with average $\avg{x}$ and standard deviation $\sigma$, we have 
\begin{align*}
  \avg{\cos\left( \frac{\dm x}{2} \right)} = \cos\left( \frac{\dm}{2} \avg{x}\right)\exp\left[-\frac{1}{2} \left(\frac{\dm}{2}\sigma\right)^2\right]
.\end{align*}.
Since the uncertainties in $L$ and $E$ are independent, we have
\begin{align*}
  \left(\frac{\sigma_{L/E}}{\avg{L/E}}\right)^2 = \left(\frac{\sigma_L}{\avg{L}}\right)^2 + \left(\frac{\sigma_E}{\avg{E}}\right)^2 
.\end{align*}

For distances $\avg{L} \le L^\text{osc}$, $\avg{P}$ oscillates as the unaveraged $P$ with a somewhat suppressed amplitude. The suppression is dependent on the width $\sigma$. For distances $\avg{L} \gg L^\text{osc}$, the oscillations are completely suppressed.
If we don't have any oscillations, the data imply an upper limit on $\avg{P}$:
\begin{align*}
  \avg{P_{\na \to \nb}} \le P^\text{max}_{\na \to \nb}
,\end{align*}
which implies an upper limit for $\sin^2 2\theta$ as a function of $\dm$, called the exclusion curve:
\begin{align*}
  \sin^2 2\theta \le \frac{2 P^\text{max}_{\na \to \nb}}{1- \avg{\cos \left( \frac{\dm L}{2E} \right)}}
.\end{align*}


\subsection*{8. Oscillations in vacuum}
Energy-momentum conservation implies that neutrinos emitted with definite energy and momenta must be produced with definite energies and momenta. Energy-momentum conservation cannot hold for different massive neutrinos and the production of superpositioned massive neutrinos is forbidden. Thus, we need to treat the neutrinos as wave packets. 

The state of the asymptotic final flavor $\nb$ is defined in terms of the state which describes an asymptotic initial state  resulting fro a weak interaction process, expressed by the $S$-matrix operator as 
\begin{align*}
  \ket{\nb} = S \ket{\na}
.\end{align*}
If the final state is a superposition of orthogonal and normalized states, it can be decomposed using the $S$-matrix coefficients $\mathcal{A}_k = \bra{\nu_{\b k}} S \ket{\na}$ 
\begin{align*}
\ket{\nb} = \sum_k \mathcal{A}_k \ket{\nu_{\b k}}
.\end{align*}
\subsection{Questions}
\begin{itemize}
  \item Do charged leptons oscillate?: The flavor of a charged lepton is defined by its mass, which does not oscillate.
  \item What are the energies and momenta?: Both the plane-wave approximation and the wave-packet treatment are independent of the values of the energies and momenta as long as the relativistic dispersion relation is satisfied and massive neutrinos are ultrarelativistic.
  \item Are wave packets necessary?: In practical applications, the oscillation probabilities are indistinguishable.
\end{itemize}
\subsection*{9. Oscillations in matter}
Neutrinos propagating in matter are subject to a potential due to the coherent forward elastic scattering with the particles in the medium. This works as a refraction index and modifie the neutrino mixing angle. The small number of incoherent scatterings with the particles in the medium can be neglected since the mean free path of solar neutrinos ($E \sim \si{1}{MeV} $) in normal matter is 0.1 light years. e can ignore incoherent cattering in Earth for $E > \si{10e5}{MeV}$. 

In matter with varying density, it is possible through the MSW mechanism to have resonant flavor transitions, when the matter modified mixing angle reache its maximum of $\pi /4$. Thi eplain hy e can have maial flavor tranition ithout a maimal vacuum miing angle.

For CC, we have the effective average Hamiltonian
\begin{align*}
  \bar{H}^\text{CC}_\text{eff}(x) = \sqrt{2}\, G_F N_e \an{eL}(x) \gamma^0 \n{eL}(x)      
\,,\end{align*}
with $V^\text{CC} = \sqrt{2}G_F N_e $. For NC, if we assume electric neutrality, the only contributing g-factors are the neutrons, so $V^\text{NC} = -\sqrt{2}/2 G_F N_n$, and the total Hamiltonian is now
\begin{align*}
  \bar{H}_\text{eff} = \sum_{\a = e, \mu, \tau} V_\a \an{\a L}(x) \gamma^0 \n{\a L}(x)
\,,\end{align*}
with the potentials
\begin{align*}
  V_\a = V_\text{CC} \delta_{\a e} + V_\text{NC} = \sqrt{2} G_F \left( N_e \delta_{\a e} - \frac{1}{2} N_n \right) 
\,.\end{align*}

The potential energy of ultrarelativistic right-handed Dirac antineutrinos have opposite sign to the potential of the left-handed neutrinos, and the potential of ultrarelativistic left-handed Dirac antineutrinos is strongly suppressed.
In the ultrarelativistic limit, neutrinos can be considered massless in interactions. The potential energy of ultrarelativistic left-handed Majorana neutrinos are the same as the potential for left-handed Dirac neutrinos. The potential energy of ultrarelativistic right-handed Majorana neutrinos is the same as that of right-handed Dirac antineutrinos.

Interactions with matter conserve helicity. 
The evolution equation for the flavor transition amplitudes is 
\begin{align*}
  i \dv{}{x}\psi_{\a \b}(x) = \sum_\eta \left( \sum_k U_{\b k} \frac{\dm_{k1}}{2E} U^*_{\eta k} + \delta _{\b e} \delta_{\eta e} V_\text{CC} \right) \psi_{\alpha \eta}(x)
.\end{align*}
And the matrix form is
\begin{align*}
  i \dv{}{x} \Psi_\a = H_F \Psi_\a, \quad H_F = \frac{1}{2E}(U \mathbb{M}^2 U^\dagger + \mathbb{A})
,\end{align*}
with
\begin{align*}
  \Psi_\a &= \begin{pmatrix} \psi_{\a e} \\ \psi_{\a \mu} \\ \psi_{\a \tau} \end{pmatrix} , \quad
  \mathbb{M}^2 = \begin{pmatrix} 0 & 0 & 0 \\
                                 0 & \\dm[21] & 0 \\
                                 0 & 0 & \dm_{31}
                 \end{pmatrix}, \\
    \mathbb{A} &= \begin{pmatrix} A_\text{CC} & 0 & 0 \\
                                 0 & 0 & 0 \\
                                 0 & 0 & 0
                 \end{pmatrix}
\,,\end{align*}
where 
\begin{align*}
  A_\text{CC} = 2EV_\text{CC} = 2 \sqrt{2}E G_F N_e 
.\end{align*}
From this we can prove that the Majorana phases are irrelevant for vacuum oscillations as well as matter oscillations.

\subsection*{MSW}
The effective mixing angle in matter is given by
\begin{align*}
\tan 2\theta_\t{M} = \frac{\tan 2\theta}{1- \frac{A_\text{CC}}{\dm \cos 2\theta}}
,\end{align*}
and is resonant when $A_\text{CC}$ becomes equal to
\begin{align*}
  A^\t{R}_\t{CC} = \dm \cos 2\theta
,\end{align*}
which corresponds to the electron number density
\begin{align*}
  N^\t{R}_\t{e} = \frac{\dm \cos 2\theta}{2\sqrt{2} E G_\t{F} }
.\end{align*}
Resonance can exist only if $\theta < \pi / 4$ for neutrinos, and only if $\theta > \pi / 4$ for antineutrinos. 

We have the evolution equation
\begin{align*}
  i \dv{}{x} \begin{pmatrix} \varphi_\t{e1} \\ \varphi_\t{e2} \end{pmatrix} 
  = \frac{1}{4E} \begin{pmatrix} -\dm_\t{M} & -4Ei \dd \theta_\t{M} / \dd x \\
                                 4Ei \dd \theta_\t{M} / \dd x & -\dm_\t{M}  
                 \end{pmatrix}
                 \begin{pmatrix} \varphi_\t{e1} \\ \varphi_\t{e2}  \end{pmatrix} 
.\end{align*} 
If matter density is constant, we recover 
\begin{align*}
  P_{\nu_{e} \to \nu_{\mu}} = \sin^2 2 \theta_{\t{M}} \sin^2 \left( \frac{\dm_{\t{M}}x}{4E} \right)
.\end{align*}
If matter density is not constant, we need to account for 
\begin{align*}
  \dv{\theta_\t{M}}{x} = \frac{1}{2} \frac{\sin 2\theta_\t{M}}{\dm_\t{M}} \dv{A_\t{CC}}{x}
.\end{align*}
The off-diagonal elements in the evolution equation generate transitions between $\n{1}^\t{M}$ and $\n{2}^\t{M}$, which can be neglected if they are much smaller than the diagonal elements. To quantify this, we introduce the adiabaticity parameter
\begin{align*}
  \gamma = \frac{\dm_\t{M}}{4E \abs{\dd \theta_\t{M} / \dd x}} = \frac{(\dm_\t{M})^2}{2E \sin 2\theta_\t{M} \abs{\dd A_\t{CC}} / \dd x}
.\end{align*}
If $\gamma \gg 1$ in all points of the trajectory, the evolution is adiabatic, and the transitions between $\nu_1^\t{M}$ and $\nu_2^\t{M}$ are negligible. In this case, the survival probability is given by
\begin{align*}
  P^\t{adiabatic}_{\ne \to \ne}(x) &= \frac{1}{2} + \frac{1}{2} \cos 2\theta_\t{M}^\t{(i)} \cos 2\theta_\t{M}^\t{(f)} \\
                                       & \quad + \frac{1}{2} \sin 2\theta_\t{M}^\t{(i)} \sin 2\theta_\t{M}^\t{(f)} \cos{\left(\int_0^x \frac{\dm_\t{M}(x')}{2E} \dd x' \right)}
,\end{align*}
where $\theta_\t{M}^\t{(f)}$ is the effective mixing angle at the detection point.
For solar or supernova neutrinos, we can set the final effective mixing angle $\thm{(f)}$ equal to the vacuum mixing angle $\theta$ since the density of the detector is too small. Thus we set the final angle to the effective angle of the propagating medium. If the travel distance is very large and $\dm_\t{M}$ not too small, the phases in the probabilities are very large and rapidly oscillating as functions of the neutrino energy. The average of the probability over the energy resolution of the detector then washes out all interference terms. 
 The distance between source and detector is huge, so the phase of the cosine is very large and has a variation much larger than $2\pi$ in the energy resolution interval of the detector. Thus, the cosine averages to zero and we have the average survival probability 
\begin{align*}
  \bar{P}^\t{adiabatic}_{\ne \to \ne} = \frac{1}{2} + \frac{1}{2}\cos{2 \theta^\t{(i)}_\t{M}}\cos{2\theta}
.\end{align*}

The maximum violation of adiabaticity (MVA) occurs when $\gamma$ is in the order of 1 or smaller, and the transitions reach their maximal size at the minimum of $\gamma$. This is in general a different point than the resonance point. Nonadiabatic transitions between $\n{1}^\t{M}$ and $\n{2}^\t{M}$ can be calculated by approximating the MVA point with the resonance point (except for solar neutrinos). 

\subsection*{Slab approximation}
When the mediums density can be approximated by a series of slabs with constant density, we have plane wave propagation with phases 

$\exp(\pm i \dm_\t{M} \Delta x / 4E)$. 
We have the flavor wave function evolution

\begin{align*}
  \Psi_e(x_n) = \sum_{i=0}^{n-1}[U_\t{M}\,\mathcal{U}_\t{M}(x_{i+1} - x_{i})\,U^\dagger_\t{M}]_{(i+1)}\Psi_e(x_i)
,\end{align*}
with the unitary evolution operator per slab
\begin{align*}
  \mathcal{U}_\t{M}(\Delta x) = \begin{pmatrix} 
    \exp(i\dm_\t{M} \Delta x / 4E) & 0 \\
                                  0 & \exp(-i\dm_\t{M} \Delta x / 4E) 
                                \end{pmatrix}
,\end{align*} 
in the slab diagonal basis. 
The unitary evolution operator in the flavor basis in each slab with constant density is given by
\begin{align*}
  \mathcal{U}(x) = \cos{\varphi (x)} - i(\vec{n} \cdot \vec{\sigma})\sin{\varphi (x)}
.\end{align*}
with the phase
\begin{align*}
  \varphi (x) = \frac{\dm_\t{M} x}{4E}
.\end{align*}

Parametric resonance occurs when we have a dynamical system whose parameters, such as the matter potential, vary periodically in time. The phase oscillations then amplifies the transition probability.

\subsection*{Three neutrino mixing}
Now, simultaneous transitions among all three flavors are allowed.
Neutrinos oscillate with two different squared-mass differences: $\dm[SOL]$ and $\dm[ATM]$ with the hierarchy $\dm[SOL] \ll \dm[ATM]$. The three massive neutrinos could be Dirac, Majorana, or be generated by a Dirac-Majorana mass term through the see-saw mechanism. We have three squared-mass differences, two of which are independent since they sum to zero. From these requirements, we can form two three-neutrino mass hierarchies. These hierarchies can allow us to approximate three-neutrino mixing to two-neutrino mixing since the effects of large and small squared-mass differences can be separated.

We have the following possible channels:
\begin{align*}
  \ne \leftrightarrow \nm\,, \quad \ne \leftrightarrow \nt\,, \quad \nm \leftrightarrow \nt
.\end{align*}

Atmospheric experiments are sensitive to the oscillations due to $\dm[31]$. Considering the hierarchy
\begin{align*}
  \dm[21] \ll \dm[31] \simeq \dm[32]
,\end{align*}
which implies that there is a dominance of the largest squared-mass difference $\dm[31]$. We then have two groups of massive neutrinos: $\{\nu_\t{1}, \nu_\t{2}\}$ and $\{\nu_\t{3}\}$. Thus, we have a two-neutrino-like effective oscillation probability
\begin{align*}
  P^\t{eff}_{\na \to \nb} = \sin^2{2 \th^\t{eff}_{\a\b}} \sin^2{\left( \frac{\dm[31]L}{4E} \right)}
.\end{align*}
All oscillation channels are open and have the same oscillation length. The probabilities are independent of the mixing angle $\th[12]$. The sign of $\dm[31]$ tells us what squared-mass hierarchy we have. 

Solar experiments are sensitive to the oscillations due to $\dm[21]$. Here we have an active small mass dominance, and oscillations due to the large squared-mass difference $\dm[31]$ are washed out. We have one neutrino per group.

If one if the elements of the mixing matrix is small, we have bilarge mixing. We can then parametrize the mixing matrix with only two mixing angles. The matrix is then real and no CP or T violations exist. We can use this case to analyze data in which $\ne$ are produced or detected.

For antineutrinos, the matter potential changes sign, so the oscillation probabilities of neutrinos and antineutrinos are different even if the fundamental Lagrangian is CP invariant. In other words, there is a matter-induce CP violation in the oscillation probabilities due to the fact that the medium is not CP-invariant. However, if the probability is independent of the CP violating phase $\delta_\t{13}$, the violation is not observable. Moreover, since the medium is also not CPT-invariant, there is a matter-induced CPT violation in the oscillation probabilities. 

\section*{Notes from~\cite{heros2020}}
\subsection{5. Standard Neutrino Oscillations}
The flavor states have extremely small cross-sections compared to all the other particles, and propagate as an admixture of "hidden" mass states. The energy expression in the time-evolution operator can be expressed as a binomial expansion and truncated, yielding the ultrarelativistic approximation $t \simeq L, p \simeq E$. We then have
\begin{align*}
  \ket{\nu_j,t} &= e^{-iE_j t}\ket{\nu_j,0} \\
                &= e^{-i\left( p + \frac{m^2_j}{2p}L \right)}\ket{\nu_j,0} \\
                &= e^{-ipL} e^{-i\left(\frac{m^2_j L}{2p} \right)}\ket{\nu_j,0} \\
                &\simeq e^{-i \frac{m^2_j L}{2E}}\ket{\nu_j,0} 
,\end{align*}   
where $\exp(-ipL)$ was be regarded as a irrelevant phase factor that disappears when calculating the probabilities. With the PMNS matrix, we now have
\begin{align*}
  \sum_j U^*_{\a j}\ket{\nu_j} = \sum_j U^*_{\a j} e^{-i \frac{m^2_j L}{2E}}\ket{\nu_j,0} 
.\end{align*}

The neut oscillations are driven by the squared-mass differences between the mass eigenstates. If $\dm_{ji} = 0$, the flavor transition probability is constant for all $E$ and all $L$. Experimental oscillation results confirm the non-zero nature of at least two of the three mass eigenstates. All in all, we have nine elements of the mixing matrix, and two squared-mass differences.

Experimental uncertainties on the mixing matrix elements create unphysical probabilities. Using the $+3\sigma$ values for all the individual elements will make some oscillation channels go below 0 and above 1. To preserve probabilities in the $[0-1]$ range, we can reparametrize the contributions of the matrix elements. By considering atmospheric and solar neutrinos separately, and considering the experiment sensitivities, we can substitute e.g. $\dm[31]$ with $\dm[32]$. We can then construct simplified transition probabilities for each channel. We make approximations such as $\cos{\th{13}} \simeq 1$, since $\th{13} \simeq 8.5 ^\circ$ if magnitude of the oscillation is driven by a larger angle, such as $\th{23}$. Thus, we can express different probability channels, having different mixing elements, with the same angles. 

If the mixing matrix is unitary, we can express the elements in terms of each other through a unitary triangle, similar to the construction of the squared-mass differences.

The mixing matrix can be decomposed as 
\begin{align*}
  &\t{LBL \& Atmospheric} \hspace{1.4 cm} \t{Reactor} \hspace{2.3 cm} \t{Solar} \\
  &\left[\begin{array}{ccc}
1 & 0 & 0 \\
0 & c_{23} & s_{23} \\
0 & -s_{23} & c_{23}
\end{array}\right]\left[\begin{array}{ccc}
c_{13} & 0 & s_{13} e^{-i \delta} \\
0 & 1 & 0 \\
-s_{13} e^{i \delta} & 0 & c_{13}
\end{array}\right]\left[\begin{array}{ccc}
c_{12} & s_{12} & 0 \\
-s_{12} & c_{12} & 0 \\
0 & 0 & 1
\end{array}\right]
\,.\end{align*}

The higher mass of the $\tau$ lepton suppresses the other pion and kaon decay channels kinematically. Thus, there is an abundance of pion and kaon decay into atmospheric $\ne$ and $\nm$. Atmospheric neutrinos has a wide range of $L / E$ combinations, which is a significant benefit for the study. This fact, together with the physical size, somewhat makes up for the (c.f accelerator experiments) poor energy and direction resolution of neutrino telescopes. The deeper a neutrino telescope is
located, the more atmospheric muons are absorbed by the material. Atmospheric neutrinos are always down-going, since they can't penetrate the whole Earth. Oscillated neutrinos are up-going. Atmospheric neutrinos emit observable light as they enter a detector, so they have an "uncontained" starting vertex at the edge of the detector. This can veto background muons, which have a "contained" starting vertex. Events reconstructed as down-going may be selected only if the starting vertex is contained and reconstructs sufficiently far from the outer edge of the detector. We can use ML to create event selections to reject background atmospheric muons while keeping neutrinos. A muon from a $\nm$ CC will produce a track-type event as opposed to a cascade for CC $\ne$ and $\nt$ or NC $\ne, \nm, \nt$ interactions. For a $\nmm$ analysis, we optimize the event selection to select samples that are rich in tracks. For analyses involving final states $\ne$ or $\nt$, we priorotize tracks and cascades.

Oscillation parameters are taken from data and MC simulations, binned into histograms of the incoming zenith angle, and compared to e.g. $\chi^2$. If the events can be sub-divided into tracks and cascade (or any other event based to event selection), we can use the feature that $\chi^2$ is the combination of $\chi^2_\t{track} + \chi^2_\t{cascade}$.
$\nt$ survival has direct measurements that relates to $\abs{U_{\tau 3}}$. The energy threshold for $\nt$ to $\tau$ lepton decay is $\sim 3.4$ GeV. Most accelerator beams have a peak $< 3$ GeV, so most of the oscillated $\nt$ from an accelerator are forbidden from interacting via CC. $\nt$ also has a smaller cross-section compared to the other neutrinos. 

Because conventional neutrino oscillation experiments measured the mass-squared difference as functions of trigonometric expressions that assume vacuum oscillations and that are insensitive to whether $\dm$ is positive or negative, the ordering of the mass eigenstates requires other methods. We can have a normal NMO ($\nu_3$ heaviest), or an inverted ($nu_3 lightest$). To probe this, we need a large sample and coherent scatterings of neutrinos with atomic electrons as the neutrinos propagate through matter. For a $50:50$ flux of neutrinos:antineutrinos composed of $\nu_{\mu,e}$, the the oscillation probabilities due to mass ordering will average out. Thus, detectors what cannot distinguish between particles and antiparticles, will have a hard time probing the NMO if we don't have a statistical significance between neutrino and antineutrino events. Since the NMO hypothesis is binary, different statistical treatments are needed, as opposed to the ones needed for the other oscillation parameters.

\subsection*{6. Sterile Neutrinos}
The scale of the BSM physics that produce neutrino oscillations is unknown. The particles involved in neutrino mass generation could have an impact in oscillations, either by direct production of sterile states or by indirect effects of non-standard interactions induced by the new particles producing neutrino masses. 

Sterile neutrinos in the eV range can account for anomalies found in short-baseline accelerator oscillation experiments. There are ways to avoid the bounds on extra relativistic neutrino species placed by cosmology (non-standard cosmology, right-handed neutrino interactions, light mediators). 
There are two possible scenarios:
\begin{itemize}
  \item Phenomenological models (PM): The sterile neutrino states do not contribute to the observes active neutrino masses. Therefore, the complete model must contain extra particles that generate such masses. It is also assumed that the new physics responsible for neutrino masses has no impact on their oscillations. Thus, we can use a model-independent phenomenological approach in which only the light neutrinos are included. We then have  a generic neutrino mass matrix of dimension $3+N_s$. For $n = 3+N_s$ neutrino flavors, we have $n(n-1)/2$ rotation angles $\th_{ij}$ and $(n-1)(n-2)/2$ physical phases.
  \item Mini-seesaw models (MM): The sterile neutrinos have for Ev Majorana masses, as well as Dirac masses that mix them with the active ones. This is the simplest extension of the standard model that can account for active neutrino masses. The masses and mixings of the sterile neutrinos are partially determined by the PMNS matrix as well as the measured neutrino mass splitting. MM has fewer free parameters than PM. Here, we have $2N_s$ free angles and $3(N_s-1)$ free phases, if $N_s \le 3$. The remaining angles and phases are not independent, and can be written in terms of the sterile neutrino Majorana masses and the Dirac masses that mix left-handed active neutrinos with the right-handed ones.
\end{itemize}

With the additions of $N_s$ sterile neutrinos, we extend the kets:
\begin{align*}
  \ket{\na}\,, \quad \a &= e,\mu,\tau,s_1,\ldots ,s_{N_s} \\
  \ket{\nu_i}\,, \quad i &= 1,2,3, \ldots, 3+N_s
\,\end{align*}
A standard parametrization of the extended U matrix consists of a series of unitary rotations $V_{ij}$ in the $i-j$ plane
\begin{align*}
  U = V_{3n}V_{2n}V_{1n} \ldots V_{34}V_{24}V_{14}U_{0}
\,,\end{align*}
where $U_0 = V_{23}V_{13}V_{12}$ contains the usual $U_\nu$ matrix in the upper left $3 \times 3$ block.
With this convention, the full matrix $U$ can be expressed as 
\begin{align*}
  U = AU_0 = \begin{pmatrix} 
              (1-\a) & \Theta \\ 
                   X & Y 
            \end{pmatrix}
            \begin{pmatrix} 
              U_\nu & 0 \\
                  0 & 1
            \end{pmatrix} 
\,,\end{align*}
with
\begin{align*}
  \a = \begin{pmatrix} 
        \a_{ee} & 0 & 0 \\
        \a_{\mu e} & \a_{\mu \mu} & 0 \\
        \a_{\tau e} & \a_{\tau \mu} & \a_{\tau \tau}
       \end{pmatrix}
\,,\end{align*}
with $a_{\beta \gamma} \ll 1$,
\begin{align}
  \label{abg}
  \alpha_{\beta \gamma} \simeq\left\{
    \begin{array}{ll}
      \frac{1}{2} \sum_{i=4}^{n}\left|U_{\beta i}\right|^{2}, & \beta=\gamma \\
      \sum_{i=4}^{n} U_{\beta i} U_{\gamma_{i}}^{*}, & \beta>\gamma \\
      0, & \gamma>\beta
    \end{array}\right.
\end{align}
The $3 \times 3$ active-light mixing sub-block is described by $N=(1-\a)U_\nu$ which is no longer unitary. $\Theta$ is a $3 \times N_s$ sub-block that mixes the active and heavy states. $R = X U_\nu$ and $Y$ sub-blocks define the mixing of the sterile with the light and heavy states, respectively. Neither $R$ nor $Y$ affect oscillations among active flavors.

The impact of sterile neutrinos on oscillations is very different depending on whether the heavy neutrinos are kinematically accessible or not.
\begin{itemize}
  \item PM: The full $(3+N_s) \times (3+N_s)$ oscillation evolution matrix $S$ has to be considered.
  \item MM: Only the $3 \times 3$ sub-matrix $N = (1-\a)U_\nu$ enters in the oscillation evolution equations, but the effect of the heavy states could still be detected due to the non-unitarity of $N$. In the limit of large mass-squared splitting, $\dm L/E_\nu \gg 1$, the oscillations are too fast to be resolved, and only the averaged-out effect is observable.  This effect is equivalent to non-unitarity to leading order in the active-heavy mixing parameters.
\end{itemize}

NC NSI are often called matter NSI since they modify neutrino propagation through matter. CC NSI are referred as production/source and detection NSI.
\subsection*{NSI and tests of Lorentz invariance}
TODO
\subsection*{Formalism}

When propagating through the Earth, $\ne, \nm$ are absorbed via charged-current interactions and redistributed via NC. $\nt$ on the other hand, is regenerated, since it produces a $\tau$ that in turn decays into a $\nt$ before losing energy, and the Earth never becomes opaque to high energy $\nt$. Additionally, secondary $\ne$ and $\nm$ are also produced after $\tau$ decay into leptonic channels. For atmospheric neutrinos however, the $\nt$ flux is negligible.

Attenuation and regeneration effects from high-energy astrophysical sources are described by a set of couples PDE's. Due to the long distance traveled, the oscillations average out and the arriving neutrinos can be treated as an incoherent superposition of mass eigenstates. For atmospheric neutrinos, this is not the case since oscillations, attenuations, and regeneration effects occur simultaneously when the neutrino beam crosses the Earth. We emply the density matrix formalism for neutrino propagation through the Earth: $\rho\left(E_{\nu}, x\right)=\nu\left(E_{\nu}, x\right) \otimes \nu\left(E_{\nu}, x\right)^{\dagger}$. We have the evolution equation:
\begin{align*}
\frac{d \rho\left(E_{\nu}, x\right)}{d x}=&-i\left[H\left(E_{\nu}, x\right), \rho\left(E_{\nu}, x\right)\right] \\
&-\sum_{\alpha} \frac{1}{2 \lambda_{\alpha}\left(E_{\nu}, x\right)}\left\{\Pi_{\alpha}\left(E_{\nu}\right), \rho\left(E_{\nu}, x\right)\right\} \\
&+\int_{E_{\nu}}^{\infty} \rho\left(E_{\nu}^{\prime}, x\right) n_{N}(x) \frac{d \sigma_{\mathrm{NC}}\left(E_{\nu}^{\prime}, E_{\nu}\right)}{d E_{\nu}} d E_{\nu}^{\prime}
\end{align*},
where $\Pi_a$ is the $\na$ projector, $\lambda_\a = 1/[n_N(x) \sigma^\t{tot}_\a(E_\nu)]$ is the attenuation length of $\na$ with $n_N(x)$ the nucleon number density in the Earth, and $\sigma^\t{tot}_\alpha(E_\nu)$ the $\na$ total (CC+NC) cross-section. The first term is oscillations, second absorption, and the third term is the redistribution of the flux due to NC interactions. 
For atmospheric neutrinos, the interplay of oscillations and Earth attenuation is described by an overall exponential suppression in the oscillated fluxes:
\begin{align*}
\phi_{\alpha}\left(E_{\nu}, \theta_{z}\right)=& \phi_{\mu}^{0}\left(E_{\nu}, \theta_{z}\right) P\left(\nu_{\mu} \rightarrow \nu_{\alpha} ; E_{\nu}, L\left(\theta_{z}\right)\right) \\
& \times \exp \left\{-\int_{0}^{L\left(\theta_{z}\right)} d x / \lambda_{\alpha}\left(E_{\nu}, x\right)\right\}
\,,\end{align*}
where $L(\th_z)$ is the baseline across the Earth in a direction with zenith angle $\th_z$.

In the presence if $N_s$ light sterile neutrinos, the $(3+N_s)\times(3+N_s)$ evolution Hamiltonian is given by
\begin{align*}
  H = H_\t{vac} + U^\dagger \t{diag}(V_e + V_\t{NC}, V_\t{NC}, V_\t{NC}, 0, \ldots, 0)U
\,,\end{align*}
where $H_\t{vac} = \t{diag}[\dm_j1/2E]$, and the second term contains the matter potentials $V_e = V_\t{CC} = \sqrt{2}G_F n_e$ and $V_\t{NC} = -\sqrt{2}G_F n_n/2$ where $n_n$ is the neutron number density. Consider only one sterile neutrino and neglect solar and atmospheric oscillation lengths (good approximation of $\mathcal{O}(\t{TeV})$). We have:
\begin{align*}
  P_{\nms} = \sin^2 2\thm \sin^2 \left( \frac{\dm[M] L}{4E_\nu} \right) 
\,\end{align*}
where the effective mass difference and matter mixing angles are given by
\begin{align*}
  \dm[M] &= \sqrt{(\dm \cos 2\th[24] - 2E_\nu V_\t{NC})^2 + (\dm \sin 2\th[24])^2}\\
  \sin 2\thm &= \frac{\dm \sin 2\th[24]}{\dm[M]}
\,\end{align*}
where the mass splitting between the heavy state and the three light ones are approximated by $\Delta m$.Even for a small vacuum mixing angle $\th[24]$ the probability $\nms$ can be enhanced through MSW resonance:
\begin{align*}
  E_\nu \sim - \cos 2\th[24] \frac{\dm}{2 V_\t{NC}}
\,\end{align*}.

Heavy sterile neutrinos lead to non-unitarity can be described by extending the three-flavor oscillation framework as :
\begin{align*}
  H = H_\t{std} + \sum_n \left(\frac{E^n}{\Lambda_n})^n \tilde{U}^\dagger_n O_n \tilde{U}_n\right)
\,\end{align*}
where the first term is the standard neutrino Hamiltonian (including the matter potential). $O_n = \t{diag}(O_{n1}, O_{n2}, O_{n3})$ and $\Lambda_n$ set the new physics scale, and $\tilde{U}_n$ stands for the mixing matrix that describes the flavor structure of the new physics. A minimal SM extension is restricted to dimension 4, and only has two terms corresponding to $n = 0,1$. NSI (and thus non-unitarity) also lead to $n=0$ terms. 
The $\nmt$ sector is a good approximation for high energies where the solar oscillation scale is very suppressed. Herem the matter potential drops out of the Hamiltonian, and we have
\begin{align*}
  H &= \frac{1}{2E}U^\dagger(\th)\begin{pmatrix} 
                                  0 & 0 \\
                                  0 & \dm[31]
                                \end{pmatrix}
                  U(\th) \\
    &+ \sum_n \sigma^\pm_n \frac{\Delta \delta_n E^n}{2}U(\xi_n) \begin{pmatrix} 1 & 0 \\ 0 & -1 \end{pmatrix}U^\dagger(\xi_n)
\,,\end{align*}  
where $\sigma^\pm_n$ accounts for the possible relative sign of the new physics contribution between neutrinos and antineutrinos. $\Delta \delta_n = 2 \Delta O_n / \Lambda ^n_n$ is the difference of the eigenvalues divided by the new physics scale, and $\xi_n$ is the mixing angle.

For this Hamiltonian, the transition probability for the $\nmt$ sector is given by
\begin{align*}
  P_{\nmt} = 1- \t{Tr}(\Pi_\mu \rho) = \sin^2 2\Theta sin^2 \left( \frac{\dm[31]L}{4E_\nu}R \right)
\,,\end{align*}
where
\begin{align*}
  R^{2}&=1+R_{n}^{2}+2R_{n}\left(\cos 2\theta_{23}\cos 2\xi_{n}+\sin 2 \theta_{23} \sin 2\xi_{n} \cos \eta_{n}\right)\\
  \sin ^{2} 2 \Theta &= \frac{1}{R^{2}}\left(\sin 2 \theta_{23}+R_{n}^{2} \sin ^{2} 2 \xi_{n}+2 R_{n} \sin 2 \theta_{23} \sin 2 \xi_{n} \cos \eta_{n}\right) \\
  R_{n}&=\sigma_{n}^{+} \Delta \delta_{n} \frac{2 E^{n+1}}{\Delta m_{31}^{2}}
\,\end{align*}
\subsection*{Searches with Neutrino Telescopes}
TODO
\subsection*{Sterile neutrino oscillations}
Sterile neutrinos produce a very fast oscillations at the low-energy part of the atmospheric neutrino flux. In the two flavor approximation, the effective Hamiltonian for the averaged oscillations is given by
\begin{align*}
  \tilde{H}_0 = \frac{\dm[31]}{4E} 
  \begin{pmatrix} -\cos(2\th_{23}) & \sin(2\th_{23}) \\ \sin(2\th_{23}) & \cos(2\th_{23}) \end{pmatrix}
  + V_\t{NC} \begin{pmatrix} 2 \a_{\mu\mu} & \a^*_{\tau\mu} \\ \a_{\tau\mu} & 2\a_{\tau\tau} \end{pmatrix} 
\,,\end{align*}
$V_\t{NC} = \sqrt{2}G_F n_n/2 $ where $V_\t{NC} = -\sqrt{2}G_F n_n/2 $ and the matrix elements $\a_{\b \gamma}$ depend on the sterile-active mixing and are defined in \eqref{abg}.

An eV mass splitting produces a large effect when neutrinos propagate in the Earth at TeV energies. So we have an enchantment due to matter effects with atmospheric neutrinos. IceCube has done this analysis for $\nm$ disappearance, and found $dm[41] = 0.92 eV^2$, $\abs{U_{\mu 4}} = 0.14$. 

If the mass scale $\dm$ is much larger than $E/L$, the oscillation effect is averaged out. Changes is the matter potential are still visible, though. This allows the usage of high-energy atmospheric data to test mass sterility above 10 eV.

Since most of the neutrinos observed in experiments travel through ordinary matter, they may be a good test of NSI neutrino-matter interactions.
\begin{itemize}
  \item Low energy: For low energies, vacuum oscillations are dominating. We are restricted to the $\nmt$ sector. The small effect of the matter potential comes from measuring distortions on the oscillation effect.
  \item High energy: The oscillation frequency in vacuum goes as $\dm/E$, and the matter potential term does not depend on the neutrino energy. Thus, the propagation of high-energy neutrinos is going to be dominated by the matter potential, and may be a great tool to test NSI.
\end{itemize}

At low energies, the telescopes are able to measure the atmospheric oscillation. In the two-flavor approximation, we have
\begin{align*}
  P_{\nmt} = \sin^2 2\Theta \sin^2 \left( \frac{\dm[31]L}{4E_\nu}R \right)  
\,,\end{align*}
with
\begin{align*}
\sin ^{2} 2 \Theta &=\frac{1}{R^{2}}\left(\sin 2 \theta_{23}+R_{0} \sin 2 \xi\right)^{2} \\
R &=\sqrt{1+R_{0}^{2}+2 R_{0} \cos 2\left(\theta_{23}-\xi\right)} \\
R_{0} &=\frac{\phi_{\operatorname{mat}}}{\phi_{\mathrm{vac}}}=V_{d} \frac{2 E}{\Delta m^{2}} \sqrt{4 \varepsilon_{\mu \tau}^{2}+\varepsilon^{2}} \\
\sin 2 \xi &=\frac{2 \varepsilon_{\mu \tau}}{\sqrt{4 \varepsilon_{\mu \tau}^{2}+\varepsilon^{\prime 2}}}
\,,\end{align*}
and the matter potential $\phi_{\mathrm{mat}}=\left(V_{d} L \sqrt{4 \varepsilon_{\mu \tau}^{2}+\varepsilon^{\prime 2}}\right) / 2$ with $V_d = \sqrt{2}G_F n_d$. 

For low energies, the effect of NSI and attenuation differs from neutrinos and antineutrinos, while at high energies, the ratio of propagated fluxes coincides. In the case in which the vacuum and matter terms in the oscillation phase are of the same order of magnitude ($\dm[31]L/4E_\nu \sim V_d \sqrt{4\varepsilon^2_{\mu \tau}} + \varepsilon^{\prime 2} $, i.e. $R_0 = \mathcal{O} (1)$) we have
\begin{align*}
  P_{\nmt} \simeq \left( \sin 2 \th[23] \frac{\dm[31]}{2e_\nu} + 2V_d \varepsilon_{\nu \tau} \right)^2 + \left( \frac{L}{2} \right) 
\,.\end{align*}

For higher energies, the matter NSI term dominates over vacuum oscillations, and for $V_\t{NSI}/L \ll 1$ we have
\begin{align*}
  P_{\nmt} \simeq (\sin 2\xi)\phi_\t{mat}^2= (\varepsilon_{\mu\tau}V_d L)^2
\,.\end{align*}

The optimal strategy is to combine the low- and high-energy samples like above to constrain the diagonal NSI parameter $\varepsilon^\prime$. 

\subsection*{Tests of Lorentz invariance}
TODO

\section*{Notes from~\cite{kopp2013}}
We do not assume a seesaw scenario, where the Dirac and Majorana mass matrices of the sterile neutrinos are the only source of neutrino mass and mixing. Instead, we consider so-called phenomenological sterile neutrino models, where the $3+s$ neutrino mass eigenvalues and the mixing parameter $U_{\a i}$ are considered to be completely independent. 
Neutrino masses in excess of a few eV may be difficult to reconcile with cosmological observations. CMB data from the PLANCK satellite puts $N_\t{eff}$ in BBN between $3.30^{+0.54}_{-0.51}$ and $3.62^{+0.50}_{-0.48}$.

The mass eigenstates $\nu_1,\ldots, \nu_{3+s}$ are labeled such that $\nu_1, \nu_2, \nu_3$ contribute mostly to the active flavor eigenstates. The mass states $\nu_4, \nu_5$ are then mostly sterile, and provide mass-squared differences in the range $\SI{0.1}{eV^2} \lesssim \abs{\dm[41]}, \abs{\dm[51]} \lesssim \SI{10}{eV^2}$. For $s=1$, we always assume $\dm[41] > 0$, and for $s=2$ we have two cases. When both $\dm[41], \dm[51]>0$, we say "3+2". When one of them is negative, we say "1+3+1".

The appearance probability in the SBL limit (effects of $\dm[21]$ and $\dm[31]$ can be neglected) for the 3+2 case:
\begin{align*}
  P_{\nu_{\alpha} \rightarrow \nu_{\beta}}^{\mathrm{SBL}, 3+2}=4\left|U_{\alpha 4}\right|^{2}\left|U_{\beta 4}\right|^{2} \sin ^{2} \phi_{41}+4\left|U_{\alpha 5}\right|^{2}\left|U_{\beta 5}\right|^{2} \sin ^{2} \phi_{51} & \\
  +8\left|U_{\alpha 4} U_{\beta 4} U_{\alpha 5} U_{\beta 5}\right| \sin \phi_{41} \sin \phi_{51} \cos \left(\phi_{54}-\gamma_{\alpha \beta}\right)
\,,\end{align*}
with 
\begin{align*}
  \phi_{i j} \equiv \frac{\Delta m_{i j}^{2} L}{4 E}, \quad \gamma_{\alpha \beta} \equiv \arg \left(I_{\alpha \beta 54}\right), \quad I_{\alpha \beta i j} \equiv U_{\alpha i}^{*} U_{\beta i} U_{\alpha j} U_{\beta j}^{*}
\,.\end{align*}
Since the probability is invariant under the transformations $4 \leftrightarrow 5$ and $\gamma_{\a \b} \to -\gamma_{\a \b}$, we can restrict the parameter range to $\dm[54] \ge 0$, or equivalently $\dm[51] \ge \dm[41]$. 
The survival probability is given by
\begin{align*}
  P_{\nu_{\alpha} \rightarrow \nu_{\alpha}}^{\mathrm{SBL}, 3+2}
  &=1-4\left(1-\sum_{i=4,5}\left|U_{\alpha i}\right|^{2}\right) \sum_{i=4,5}\left|U_{\alpha i}\right|^{2} \sin ^{2} \phi_{i 1} \\
  &-4\left|U_{\alpha 4}\right|^{2}\left|U_{\alpha 5}\right|^{2} \sin ^{2} \phi_{54}
\,.\end{align*}
We also have the LBL approximations in the limit $\th[41], \th[51], \th[54] \to \infty$ and $\th[21] \to 0 (\dm[21] \to 0)$:
\begin{align*}
  P_{\nu_{\alpha} \rightarrow \nu_{\beta}}^{\mathrm{LBL}, 3+2}
  &= 4\left|U_{\alpha 3}\right|^{2} \abs{U_{\beta 3}}^{2} \sin ^{2} \phi_{31}+2 \sum_{i=4}^{5}\left|U_{\alpha i}\right|^{2}\left|U_{\beta i}\right|^{2}\\
  &+2 \operatorname{Re}\left(I_{\alpha \beta 45}\right)+4 \operatorname{Re}\left(I_{\alpha \beta 43}+I_{\alpha \beta 53}\right) \sin ^{2} \phi_{31} \\
  &+2 \operatorname{Im}\left(I_{\alpha \beta 43}+I_{\alpha \beta 53}\right) \sin \left(2 \phi_{31}\right)
\,,\end{align*}
and the survival probability
\begin{align*}
  P_{\nu_{\alpha} \rightarrow \nu_{\alpha}}^{\mathrm{LBL}, 3+2}
  &= \left(1-\sum_{i=3}^{5}\left|U_{\alpha i}\right|^{2}\right)^{2}+\sum_{i=3}^{5}\left|U_{\alpha i}\right|^{4} \\
  &+2\left(1-\sum_{i=3}^{5}\left|U_{\alpha i}\right|^{2}\right)\left|U_{\alpha 3}\right|^{2} \cos \left(2 \phi_{31}\right)
\,,\end{align*}

We use the following parametrization for U when $n = 3+s = 5$:
\begin{align*}
  U=V_{35} O_{34} V_{25} V_{24} O_{23} O_{15} O_{14} V_{13} V_{12}
\,,\end{align*}
where $O_{ij}$ represents a real rotation matrix by an angle $\th_{ij}$ in the $i-j$ plane, and $V_{ij}$ represents a complex rotation by an angle $\th_{ij}$ and a phase $\varphi_{ij}$. $V_{45}$ mixes sterile states and is unobservable and not included. 

Disappearance experiments in the $\ne$ sector probe $\abs{U_{ei}}$. In the SBL limit of the 3+1 scenario, the only relevant parameter is $\abs{U_{e4}}$. For 3+2, $\abs{U_{e5}}$ too.

\begin{align*}
  P_{e e}^{\mathrm{SBL}, 3+1}
  &=1-4\left|U_{e 4}\right|^{2}\left(1-\left|U_{e 4}\right|^{2}\right) \sin ^{2} \frac{\Delta m_{41}^{2} L}{4 E} \\
  &=1-\sin ^{2} 2 \theta_{e e} \sin ^{2} \frac{\Delta m_{41}^{2} L}{4 E}
\,,\end{align*}
where the effective $\ne$ mixing angle is 
\begin{align*}
  \sin^2 2\theta_{ee} = 4 \abs{U_{e4}}^2(1 - \abs{U_{e4}^2})
\,,\end{align*}

\section*{Articles}
\subsection*{Notes from https://inspirehep.net/files/b82d969033bc053e64b4b90b3806ea7f}

A neutrino produced at a source in association with a charged lepton $\ell_\a$ is simply
\begin{align*}
  \ket{\na^s} = \ket{\na}
\,,\end{align*}
and a neutrino that produces a charged lepton $\ell_\b$ at a detector is
\begin{align*}
  \bra{\nb^d} = \bra{\nb}
\,.\end{align*}
The propagation is governed by the time-evolution equation
\begin{align*}
  i \dv{}{t} \begin{pmatrix} 
                \ne \\ \nm \\ \nt
             \end{pmatrix}
  = \frac{1}{2E} 
  \left[ 
    U^\dagger \begin{pmatrix} 
                0 & 0 & 0 \\
                0 & \dm[21] & 0 \\
                0 & 0 & \dm[31]
              \end{pmatrix} 
    U +
    \begin{pmatrix} 
      A & 0 & 0 \\
      0 & 0 & 0 \\
      0 & 0 & 0
    \end{pmatrix}
  \right]
  \begin{pmatrix} 
    \ne \\ \nm \\ \nt
  \end{pmatrix}
\,,\end{align*}
with the matter potential $A = 2\sqrt{2}G_F n_e E $.

BSM, it's possible to have CC-like operators that affect the interactions of neutrinos with charged leptons. If thee operator are not diagonal in the flavor basis, then the production and the detection of neutrinos are affected. Thus, the neutrino state produced at the source in association with the charged lepton $\ell_\a$ then also has components of the other flavor:
\begin{align*}
  \ket{\na^s} &= \ket{\na} + \sum_{\gamma = e,\,\mu,\,\tau} \varepsilon_{\a \gamma}^s \ket{\nu_\gamma} \\
  \bar{\nb^d} &= \bar{\nb} + \sum_{\gamma = e,\,\mu,\,\tau} \varepsilon_{\gamma \b}^d \bra{\nu_\gamma}
\,.\end{align*}
The matrices $\varepsilon^s = (\varepsilon_{\a \gamma}^s)$ and $\varepsilon^d = (\varepsilon^d_{\gamma \beta})$ are complex with 18 real parameters each.

The NC-like operators affect the matter potential:
\begin{align*}
  i \dv{}{t} \begin{pmatrix} 
                \ne \\ \nm \\ \nt
             \end{pmatrix}
  &= \frac{1}{2E} 
  \left[ 
    U^\dagger \begin{pmatrix} 
                0 & 0 & 0 \\
                0 & \dm[21] & 0 \\
                0 & 0 & \dm[31]
              \end{pmatrix} 
  U + A
    \begin{pmatrix} 
      1 + \varepsilon_{ee}^m & \varepsilon_{e\mu}^m & \varepsilon_{e\tau}^m   \\
     \varepsilon_{\mu e}^m   &\varepsilon_{\mu \mu}^m   &\varepsilon_{\mu \tau}^m   \\
     \varepsilon_{\tau e}^m   &\varepsilon_{\tau \mu}^m   &\varepsilon_{\tau \tau}^m  
    \end{pmatrix}
  \right]
  \begin{pmatrix} 
    \ne \\ \nm \\ \nt
  \end{pmatrix}
\,.\end{align*}
For the Hamiltonian to be Hermitian, we require $\varepsilon_{\a\b}^m = \varepsilon_{\b\a}^{m\,*} $.

\subsection*{Notes from 1207.4765}

The probability of observing oscillations is large when $\dm \sim E/L$. For signals in the $\dm \sim \SI{1}{ev^2}$ range, the experiments are therefore designed with $E/L \sim \SI{1}{GeV/km}$. We call these experiments with sensitivity to $\dm \sim \SI{1}{eV^2}$ oscillations "short baseline experiments". In the wave where $E/L \ll \SI{1}{GeV/km}$, the oscillations will be rapid. When $\dm \sim \SI{1}{eV^2}$, the sensitivity to the mass splitting is lost because the oscillation probability averages to only depends on the mixing angle. Thus, "long baseline experiments" probe the mixing angle, but not $\dm$.

If the mass scales are quite different ($m_3 \gg m_2 \gg m_1$), the oscillation phenomena decouple, and the two neutrino mixing model is a good approximation in limited regions. 

The short baseline approximation assumes that the tree lowest states $\nu_1, \nu_2, \nu_3$ have masses so small as to be effectively degenerate with zero mass, more explicitly: $\nu_1 =\simeq \nu_2 \simeq \nu_3 = 0$ Thus, it reduces the for to two-, three-, and four-neutrino mass models, corresponding to (3+1), (3+2), and (3+3), respectively. SBL experiments have no $\nt$ sensitivity, so we assume $\abs{U_{\tau i}} = 0$.

In matter, the $\ne$ experiences both CC and NC elastic forward-scattering. $\nm,\nt$ only experience NC forward-scattering.

The data fall into to types:
\begin{itemize}
  \item Disappearance: the active flavor is assumed to have oscillated into a sterile neutrino and/or another flavor which is kinematically not allowed to interact.
  \item Appearance: The transition is between active flaors, but with mass splittings correpsonding to the mostly-sterile states.
\end{itemize}

\subsection*{Notes from 1202.1024}
Setting $\dm[32] \simeq \dm[21] \simeq 0$ and $\dm[41] \simeq \dm[43] \simeq \dm[42]$, and the orthogonality relations $U_{\mu1} U_{e1}^* = -U_{\mu 4}U_{e4}^*$,$\abs{U_{e1}}^2 = 1-\abs{U_{e4}}^2$, and $\sin^2 2\th_{\mu e} = 4 \abs{U_{e4}}^2 \abs{U_{\mu 4}}^2$ we have the $\ne$ appearance probability
\begin{align*}
  P_{\nme} = \sin^2 3\th_{\mu e} sin^2(1.267\dm[41] L/E)
\,,\end{align*}
and the survival probability
\begin{align*}
  P_{\nee} &= 1- \sin^2 2\th_{ee} \sin^2(1.27 \dm[41 L/E]) \\
           &= 1 - 4\abs{U_{e4}}^2(1- \abs{U_{e4}}^2)\sin^2(1.27\dm[41]L/E)
\,.\end{align*}

We can assume that the matter potential $V_s$ experienced by $\ns$ is much larger in amplitude than the standard model matter effect potentials $V_\t{CC}, V_\t{NC}$. Thus,
\begin{align*}
  V = \begin{pmatrix}
    V_\t{CC} + V_\t{NC} & 0 & 0 & 0 \\
    0 & V_\t{NC} & 0 & 0  \\
    0 & 0 & V_\t{NC} & 0  \\
    0 & 0 & 0 & V_s 
  \end{pmatrix}
  \simeq
  \begin{pmatrix}
    0 & 0 & 0 & 0\\
    0 & 0 & 0 & 0\\
    0 & 0 & 0 & 0\\
    0 & 0 & 0 & V_s
  \end{pmatrix}
\,.\end{align*}
We can also assume that the active masses in vaccum are degenerate and negligible. Thus, $\dm[41] = \dm$. Also assume that $m^2_{2,3}/2E$ are negligible relative to $V_s$. We then have the effective Hamiltonian
\begin{align*}
  H_\t{m} = \frac{\dm}{2E}\begin{pmatrix}
    U_{e4}U^*_{e4} & U_{e4}U^*_{\mu4} & U_{e4}U^*_{\tau4} & U_{e4}U^*_{s4} \\
    U_{\mu4}U^*_{e4} & U_{\mu4}U^*_{\mu4} & U_{\mu4}U^*_{\tau4} & U_{\mu4}U^*_{s4} \\
    U_{\tau4}U^*_{e4} & U_{\tau4}U^*_{\mu4} & U_{\tau4}U^*_{\tau4} & U_{\tau4}U^*_{e4} \\
    U_{s4}U^*_{e4} & U_{s4}U^*_{\mu4} & U_{s4}U^*_{\tau4} & U_{s4}U^*_{s4} + 2EV_s/\dm \\
  \end{pmatrix}
\,.\end{align*}
The eigenvalues can be reduced by assuming unitarity together
\begin{align*}
    \lambda_{1}=0 \\
    \lambda_{2}=0 \\
    \lambda_{3}=\frac{1}{4 E}\left(2 E V_{s}+\Delta m^{2}-\sqrt{\left(2 E V_{s}+\Delta m^{2}\right)^{2}-8 E V_{s} \Delta m^{2}\left(1-\left|U_{s 4}\right|^{2}\right)}\right) \\
    \lambda_{4}=\frac{1}{4 E}\left(2 E V_{s}+\Delta m^{2}+\sqrt{\left(2 E V_{s}+\Delta m^{2}\right)^{2}-8 E V_{s} \Delta m^{2}\left(1-\left|U_{s 4}\right|^{2}\right)}\right)
\,.\end{align*}.
If the active favlor content of the fourth mass eigenstate is small, we can approximate 
\begin{align*}
  1 \abs{U_{s4}}^2 \simeq 0
\,,\end{align*}
and all eigenvalues except $\lambda_4 = 1/2E(2EV_s + \dm)$ become zero, implying one effective $\dm_M$, which corresponds to 
\begin{align*}
  \dm_M = \dm + 2EV_s
\,.\end{align*}

By requiring $\abs{U_{e4}}^2 < 0.05$ and $\abs{U_{\mu 4}}^2 < 0.05$, we keep unitarity of the $3 \times 3$ matrix at the $5\%$ level. To keep $L/E \sim \SI{1}{eV^2}$, we can let $A_s$ be in the range \SIrange{10e-3}{10e-9}{eV}

\section*{Diagonalizing H}
\subsection*{2 gen}
\begin{align*} 

\,.\end{align*}


\begin{align*}
  H_{M1} &= - \frac{\sqrt{A_{cc}^{2} + 4 A_{cc} \dm[21] \sin^{2}{\left(\theta_{12} \right)} - 2 A_{cc} \dm[21] + (\dm[21])^{2}}}{4 E} \\
  H_{M2,3} &=0 \\
  H_{M4} &= \frac{\sqrt{A_{cc}^{2} + 4 A_{cc} \dm[21] \sin^{2}{\left(\theta_{12} \right)} - 2 A_{cc} \dm[21] + (\dm[21])^{2}}}{4 E}
\,.\end{align*}

\bibliographystyle{apalike}
\bibliography{notes} 
\end{document}