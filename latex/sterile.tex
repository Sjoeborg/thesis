\documentclass[twocolumn]{article}
\usepackage{physics,amsmath, amsfonts, siunitx, amssymb}
\usepackage[utf8]{inputenc}
\newcommand{\n}[1]{\ensuremath{\nu_{#1}}}
\newcommand{\na}{\ensuremath{\nu_\alpha}}
\newcommand{\nb}{\ensuremath{\nu_\beta}}
\newcommand{\ana}{\ensuremath{\bar{\nu}_\alpha}}
\newcommand{\an}[1]{\ensuremath{\bar{\nu}_{\text{#1}}}}
\newcommand{\anb}{\ensuremath{\bar{\nu}_\beta}}
\renewcommand{\a}{\ensuremath{\alpha}}
\renewcommand{\b}{\ensuremath{\beta}}
\newcommand{\ab}{\ensuremath{\alpha\beta}}

\renewcommand{\ne}{\ensuremath{\nu_e}}
\newcommand{\ns}{\ensuremath{\nu_s}}

\newcommand{\nee}{\ensuremath{\nu_e \to \nu_e}}
\newcommand{\nem}{\ensuremath{\nu_e \to \nu_\mu}}
\newcommand{\net}{\ensuremath{\nu_e \to \nu_\tau}}
\newcommand{\nes}{\ensuremath{\nu_e \to \nu_s}}

\newcommand{\nme}{\ensuremath{\nu_\mu \to \nu_e}}
\newcommand{\nmm}{\ensuremath{\nu_\mu \to \nu_\mu}}
\newcommand{\nmt}{\ensuremath{\nu_\mu \to \nu_\tau}}
\newcommand{\nms}{\ensuremath{\nu_\mu \to \nu_s}}


\newcommand{\nte}{\ensuremath{\nu_\tau \to \nu_e}}
\newcommand{\ntm}{\ensuremath{\nu_\tau \to \nu_\mu}}
\newcommand{\ntt}{\ensuremath{\nu_\tau \to \nu_\tau}}
\newcommand{\nts}{\ensuremath{\nu_\mu \to \nu_s}}

\newcommand{\Pee}{\ensuremath{P_{e  e}}}
\newcommand{\Pem}{\ensuremath{P_{e  \mu}}}
\newcommand{\Pet}{\ensuremath{P_{e  \tau}}}
\newcommand{\Pes}{\ensuremath{P_{e  s}}}

\newcommand{\Pme}{\ensuremath{P_{\mu  e}}}
\newcommand{\Pmm}{\ensuremath{P_{\mu\mu}}}
\newcommand{\Pmt}{\ensuremath{P_{\mu  \tau}}}
\newcommand{\Pms}{\ensuremath{P_{\mu  s}}}


\newcommand{\Pte}{\ensuremath{P_{P_{\tau e}}}}
\newcommand{\Ptm}{\ensuremath{P_{\tau  \mu}}}
\newcommand{\Ptt}{\ensuremath{P_{\tau  \tau}}}
\newcommand{\Pts}{\ensuremath{P_{\mu  s}}}

\newcommand{\Paeae}{\ensuremath{P_{\bar{e}  \bar{e}}}}
\newcommand{\Paeam}{\ensuremath{P_{\bar{e}  \bar{\mu}}}}
\newcommand{\Paeat}{\ensuremath{P_{\bar{e}  \bar{\tau}}}}
\newcommand{\Paeas}{\ensuremath{P_{\bar{e}  \bar{s}}}}

\newcommand{\Pamae}{\ensuremath{P_{\bar{\mu}  \bar{e}}}}
\newcommand{\Pamam}{\ensuremath{P_{\bar{\mu}  \bar{\mu}}}}
\newcommand{\Pamat}{\ensuremath{P_{\bar{\mu}  \bar{\tau}}}}
\newcommand{\Pamas}{\ensuremath{P_{\bar{\mu}  \bar{s}}}}


\newcommand{\Patae}{\ensuremath{P_{\bar{\tau}  \bar{e}}}}
\newcommand{\Patam}{\ensuremath{P_{\bar{\tau}  \bar{\mu}}}}
\newcommand{\Patat}{\ensuremath{P_{\bar{\tau}  \bar{\tau}}}}
\newcommand{\Patas}{\ensuremath{P_{\bar{\mu}  \bar{s}}}}

\renewcommand{\th}[1][]{%
  \theta\ifx\\#1\\\else_\text{#1}\fi
}
\newcommand{\thm}[1][]{%
  \theta^\text{M}\ifx\\#1\\\else_\text{#1}\fi
}
\renewcommand{\t}[1]{\ensuremath{\text{{#1}}}}
\newcommand{\avg}[1]{\ensuremath{\left\langle {#1} \right \rangle}}
\newcommand*{\dm}[1][]{%
  \Delta m^2\ifx\\#1\\\else_\text{#1}\fi
}
\newcommand{\zreco}{\ensuremath{\cos{(\theta_z^{reco})}}}
\newcommand{\ztrue}{\ensuremath{\cos{(\theta_z^{true})}}}
\newcommand{\emm}{\ensuremath{\epsilon_{\mu\mu}}}
\newcommand{\emt}{\ensuremath{\epsilon_{\mu\tau}}}
\newcommand{\eet}{\epsilon_{e\tau}}
\newcommand{\eem}{\epsilon_{e\mu}}
\newcommand{\ett}{\ensuremath{\epsilon_{\tau\tau}}}
\newcommand{\ep}{\ensuremath{\epsilon^\prime}}
\renewcommand{\ne}{\nu_e}
\newcommand{\nm}{\nu_\mu}
\newcommand{\nt}{\nu_\tau}
\newcommand{\ane}{\bar\nu_e}
\newcommand{\anm}{\bar\nu_\mu}
\newcommand{\ant}{\bar\nu_\tau}
\usepackage{physics,amsmath, amsfonts, siunitx, amssymb,cite}
\usepackage[utf8]{inputenc}
\usepackage{xr-hyper}
%\usepackage{hyperref}
\externaldocument{SM}

\begin{document}
\section{Sterile and Massive Neutrinos}\label{ch:sterile}
As we saw in Eq.~\ref{eq:j_CC}
, the neutrino fields $\nu_\alpha$ only couple to the associated lepton fields $\ell_\alpha$, conserving the lepton number $L_\alpha$. We will now introduce two separate extensions to this part of the Standard Model.

We introduce a right-handed neutrino field, $\nu_R$. It has the usual properties of the conventional left-handed neutrino such as hypercharge and color zero. Moreover, since the electroweak gauge group $\t{SU}(2)_L$ only couple to left-handed particles and right-handed antiparticles, it transforms as a singlet under the SM symmetry group $\mathrm{SU}(3)_{\mathrm{C}} \times \mathrm{SU}(2)_{L} \times \mathrm{U}(1)_{Y}$. This neutrino is \emph{sterile} since it doesn't participate any of the SM interactions. 

We extend the SM by adding a right-handed component to the Higgs-lepton Yukawa Lagrangian from Eq.~\ref{eq:YukawaLagrangian} with neutrino Yukawa couplings $Y_{\alpha \beta}^{\prime \nu}$ ,

\begin{align}
    \mathcal{L}_{H}=-\left( \frac{v + H}{\sqrt{2}} \right) \left[\ell_{\alpha L}^{\prime} Y_{\alpha \beta}^{\prime \ell} \ell_{\beta R}^{\prime} + \nu_{\alpha L}^{\prime} Y_{\alpha \beta}^{\prime \nu} \nu_{\beta R}^{\prime}\right]
\end{align}
Similar to how we diagonalized the lepton Yukawa couplings $Y_{\alpha \beta}^{\prime \ell}$ in Eq.~~\ref{eq:leptonYukawaDiag}, we diagonalize $Y_{\alpha \beta}^{\prime \nu}$ as
\begin{align}
    V_{\alpha k L}^{\nu \dagger} Y^{\prime \nu}_{\alpha \beta} V_{\beta j R}^{\nu}=Y^{\nu}_{kj} \,.
\end{align}
Now, let the neutrino field with chriality $X$ be denoted $n_X$, with components Latin numerals to distinguish them from the flavour components, i.e 
\begin{align}
    \nu_{k X} =  V_{kj X}^{\nu \dagger} \nu_{jX}^\prime\,.
\end{align}
The diagonalized Lagrangian now takes the form 
\begin{align}
    \mathcal{L}_{H} &= -\left( \frac{v + H}{\sqrt{2}} \right) \left[\ell_{\alpha L}^{\prime} Y_{\alpha \beta}^{\prime \ell} \ell_{\beta R}^{\prime} + \nu_{\alpha L}^{\prime} Y_{\alpha \beta}^{\prime \nu} \nu_{\beta R}^{\prime}\right] \nonumber \\
    &= -\left( \frac{v + H}{\sqrt{2}} \right) \left[\ell_{\alpha L}^{\prime} V_{\alpha \beta L}^{\ell} Y_{\alpha \beta}^{ \ell} V_{\alpha \beta R}^{\ell \dagger} \ell_{\beta R}^{\prime}\right.\\
    &\hspace{5.7em}+\left. \nu_{\alpha L}^{\prime} V_{\alpha k L}^{\nu} Y_{kj}^{\nu} V_{\beta j  R}^{\nu \dagger} \nu_{\beta R}^{\prime}\right] \nonumber \\
    &= -\left( \frac{v + H}{\sqrt{2}} \right) \left[\ell_{\alpha L}^\dagger Y_{\alpha \beta}^{ \ell} \ell_{\beta R} + \nu_{k L}^{\dagger} Y_{kj}^{ \nu} \nu_{j R}\right] \nonumber \\
    &= -\left( \frac{v + H}{\sqrt{2}} \right) \left[\bar{\ell}_{\alpha L} Y_{\alpha \beta}^{ \ell} \ell_{\beta R} + \bar{\nu}_{k L} Y_{kj}^\nu \nu_{j R}\right]
\end{align}
Now using the fact that $Y^\nu_{kj}$ is diagonal, we write it as $Y_{k j}^{\nu}=y_{k}^{\nu} \delta_{k j}$, leaving the Lagrangian as 
\begin{align}
    \mathcal{L}_{H} 
    &=-\left( \frac{v + H}{\sqrt{2}} \right) \left[\bar{\ell}_{\alpha L} y_{\alpha}^{\ell} \delta_{\alpha \beta} \ell_{\beta R} + \bar{\nu}_{k L} y_{k}^{\nu} \delta_{k j} \nu_{j R}\right] \nonumber \\
    &=-\left( \frac{v + H}{\sqrt{2}} \right) \left[\bar{\ell}_{\alpha L} y_{\alpha}^{\ell}  \ell_{\alpha R} + \bar{\nu}_{k L} y_{k}^{\nu} \nu_{k R}\right] \nonumber \\
    &=-\left( \frac{v + H}{\sqrt{2}} \right) \left[ y_{\alpha}^{\ell}  \bar{\ell}_{\alpha L}\ell_{\alpha R} +  y_{k}^{\nu}\bar{\nu}_{k L} \nu_{k R}\right] 
\end{align}

Now, the Dirac neutrino field is
\begin{align}
    \nu_k = \nu_{kL} + \nu_{kR}\,.
\end{align}
Multiplying $\nu_k$ with its conjugate $\bar{\nu}_k$, we get 
\begin{align}
    \bar{\nu}_k \nu_k 
    & = \bar{\nu}_{k L} \nu_{k L} +\bar{\nu}_{k R}\nu_{k L} + \bar{\nu}_{k L}\nu_{k R} + \bar{\nu}_{k R}\nu_{k R} \nonumber \\
    & = \bar{\nu}_{k L}\nu_{k R} + \bar{\nu}_{k R}\nu_{k L} \nonumber \\
    & = \bar{\nu}_{k L}\nu_{k R} + \t{h.c.}
\end{align}
The same calculation for the charged lepton field yields the same result for $\ell_k$. Substituting this result and expanding the Higgs VEV into the fields gives us
\begin{align}
    \mathcal{L}_{H} 
    &=-\left( \frac{v + H}{\sqrt{2}} \right) \left[ y_{\alpha}^{\ell}   \bar{\ell}_\alpha \ell_\alpha  +  y_{k}^{\nu} \bar{\nu}_k \nu_k \right] \nonumber \\
    &=- \frac{y_{\alpha}^{\ell} v}{\sqrt{2}}   \bar{\ell}_\alpha \ell_\alpha   -  \frac{ y_{k}^{\nu} v}{\sqrt{2}} \bar{\nu}_k \nu_k  - \frac{y_{\alpha}^{\ell}}{\sqrt{2}}   \bar{\ell}_\alpha \ell_\alpha H  -  \frac{ y_{k}^{\nu}}{\sqrt{2}} \bar{\nu}_k \nu_k H\,.
\end{align}
Thus, this extension to the SM generates neutrino masses by the Higgs mechanism, in the same fashion as with the charged leptons and the quarks:
\begin{align}
    m_k = \frac{y_k^\nu v}{\sqrt{2}}
\end{align}
%\bibliographystyle{apalike}
%\bibliography{ref}
\end{document}