\section{The Standard Model}\label{ch:SM}
In order to describe the three quantizable forces of nature, we gather the mediators of each force -- the vector bosons -- into local (gauge) symmetry groups. 
Each vector boson has one corresponding generator, the set of which constitutes the group.
The strong charge is mediated by eight massless gluons, which correspond to the eight independent generators of $\text{SU}(3)_C$. 
The weak charge is mediated by the three massive gauge bosons $W^\pm$ and $Z$, and the massless photon $\gamma$, 
which constitute the generators of $\text{SU}(2)_L$ and $\text{U}(1)_Y$. 

The subscript of each group denotes by which mechanism that force is mediated. The gluons mediate the strong force through interactions of color,
emphasized with subscript $C$. The weak force only sees left-handed particles, 
which we distinguish with the subscript $L$. And the electroweak interaction that a particle undergoes is determined by its hypercharge $Y$. 
For example, the quarks all have a non-zero color and non-zero hypercharge, 
so they participate in the strong and electromagnetic interactions. If a quark is left-handed, it will also feel the weak interaction. 
The neutrinos, on the other hand, have neither charge nor color, 
so they are invisible to both the strong and electromagnetic force. We express this by letting their fields transform as singlets under those symmetry groups.

\subsection{Beyond the Standard Model}
Together, these three interactions make up the Standard Model gauge group $\mathrm{SU}(3)_{\mathrm{C}} \times \mathrm{SU}(2)_{L} \times \mathrm{U}(1)_{Y}$. 
This determines the form of the three coupling constants, which numerical values must be experimentally measured. 
Since the vector bosons are represented by the generators, they are uniquely determined by the symmetry group. However, the scalar boson(s) and fermions are free as long as they
belong to representations of the symmetry group. By this construction, modifications to fermions rather than bosons are generally easier to make, allowing us to propose amendments to the model. 
We will use this leniency to will examine the possibility of adding a new completely new interaction, which manifests itself as modifications to the neutrino-matter interactions.
